\documentclass[executivepaper]{article}

\usepackage{graphicx}
\usepackage[utf8]{inputenc}
\usepackage[T1]{fontenc}
\usepackage[spanish]{babel} % Establece el idioma español
\usepackage{csquotes} % Carga el paquete csquotes
\usepackage{graphicx} % Required for inserting images
\usepackage{listings}
\usepackage{xcolor}
\usepackage{hyperref}
\usepackage[left=1.00cm, right=1.00cm, top=2.00cm, bottom=2.00cm]{geometry}
\usepackage{tikz}
\usetikzlibrary{shapes,arrows}
\usetikzlibrary{positioning}
\setlength{\parindent}{0.5in}
\usepackage{setspace}
\doublespacing

\lstset{
    inputencoding=utf8,
    language=Java,
    basicstyle=\ttfamily,
    columns=fullflexible
}

% Define colores para el código
\definecolor{codegreen}{rgb}{0,0.6,0}
\definecolor{codegray}{rgb}{0.5,0.5,0.5}
\definecolor{codepurple}{rgb}{0.58,0,0.82}
\definecolor{backcolour}{rgb}{0.95,0.95,0.92}

% Configuración de lstlisting
\lstdefinestyle{mystyle}{
    backgroundcolor=\color{backcolour},   
    commentstyle=\color{codegreen},
    keywordstyle=\color{magenta},
    numberstyle=\tiny\color{codegray},
    stringstyle=\color{codepurple},
    basicstyle=\ttfamily\footnotesize,
    breakatwhitespace=false,         
    breaklines=true,                 
    captionpos=b,                    
    keepspaces=true,                 
    numbers=left,                    
    numbersep=5pt,                  
    showspaces=false,                
    showstringspaces=false,
    showtabs=false,                  
    tabsize=2
}

% Configuración del paquete hyperref
\hypersetup{
    colorlinks=true,
    linkcolor=black,
    filecolor=magenta,      
    urlcolor=gray,
}

\lstset{style=mystyle}

\renewcommand{\baselinestretch}{1.5}

\newtheorem{propo}{Proposición}[section]
\newtheorem{lema}[propo]{Lema}
\newtheorem{teo}[propo]{Teorema}
\newtheorem{coro}[propo]{Corolario}
\newtheorem{defi}[propo]{Definición}
\newtheorem{obs}[propo]{Observación}
\newtheorem{ejemplo}[propo]{Ejemplo}

\newcommand{\Al}{(\mathcal{A},\mathds{F},\odot)}
\newcommand{\A}{\mathcal{A}}
\newcommand{\B}{\mathcal{B}}
\newcommand{\D}{\mathcal{D}}
\newcommand{\C}{\mathcal{C}}
\newcommand{\I}{\mathcal{I}}
\newcommand{\J}{\mathcal{J}}
\newcommand{\R}{\mathds{R}}
\newcommand{\N}{\mathds{N}}
\newcommand{\fu}{f:D\longrightarrow \mathds{R}}
\newcommand{\fun}{f:[a,b]\longrightarrow \mathds{R}}
\newcommand{\E}{\mathcal{E}}
\newcommand{\F}{\mathds{F}}
\newcommand{\op}{``}
\newcommand{\cl}{''}
\newcommand{\po}{^}
\newcommand{\X}{\mathbf{X}}
\newcommand{\Q}{\matbbb{Q}}

\title{Definiciones inductivas}
\author{Kevin Cárdenas}

\begin{document}
\begin{titlepage}
    \begin{center}
        {\Huge \textbf{Definiciones Inductivas y Coinductivas}}
        \\[9cm]
        
        \large\emph{Autor:}\\
        Kevin Cárdenas.
        \\[9cm]
        \large\emph{Ascesor:}
        Juan Carlos Agudelo.\\
        Trabajo de grado en modalidad de monografía para el titulo como matemático.\\
        2023
    \end{center}
\end{titlepage}

\newpage
\tableofcontents
\newpage
\section{Definiciones Inductivas y Coinductivas}
\subsection{¿Qué es una definición inductiva?}
\begin{defi}
\textbf{Regla, $\phi-$cerrado, Conjunto inductivamente definido}
\begin{enumerate}
\item Una regla es un par $(A,x)$, donde $A$ es un conjunto, llamado \textbf{el conjunto de premisas}, y $x$ es la conclusión.
\item Si $\phi$ es un conjunto de reglas, entonces un conjunto $A$ es $\phi - \textbf{cerrado}$ si cada regla de $\phi$ tiene sus premisas en $A$.
\item Dado $\phi$ un conjunto de reglas, definimos $I(\phi)$ como el conjunto inductivamente definido por $\phi$, que se define como:
$$L(\phi) = \bigcap\{A \, : A\,\, \phi-cerrado\}$$
\end{enumerate}
\end{defi}
Esta definición establece que una regla es un par que consiste en un conjunto de premisas y una conclusión. Un conjunto $A$ es $\phi - \textbf{cerrado}$ si cada regla de $\phi$ tiene sus premisas en $A$. Finalmente, se define el conjunto inductivamente definido por $\phi$ como el menor conjunto que contiene todos los elementos del conjunto de premisas de $\phi$ y que es cerrado bajo $\phi$. Este conjunto se define como la intersección de todos los conjuntos $\phi-$cerrados.

\begin{ejemplo}\textbf{$\mathbf{N}$ Como conjunto inductivamente definido}\\
Desde la definición que hemos dado, podemos definir los números naturales $\mathrm{N}$ como un conjunto inductivamente definido. Para ello, definimos $\phi$ como un conjunto de reglas que contienen las siguientes dos reglas:
\begin{enumerate}
\item $(\emptyset,0)$, donde $0$ es un número natural.
\item $({n},s(n))$, donde $n$ es un número natural y $s(n)$ es el sucesor de $n$.
\end{enumerate}
De ahí, es claro ver que $\mathrm{N} = I(\phi)$, pues $\mathrm{N}$ es el menor conjunto que contiene a $0$ y es cerrado bajo la operación sucesor.
\end{ejemplo}

\begin{defi}
\textbf{$\phi-$prueba}\\
$a_1, a_2, \ldots, a_n$ es una $\phi-$prueba de $b$ si
\begin{enumerate}
\item $a_n = b$
\item si $m \leq n$, entonces existe $X\subseteq \{a_i\}_{i<m}$ tal que $\phi : X \rightarrow a_m$
\end{enumerate}
Si $x$ tiene una $\phi-$prueba, entonces decimos que $x$ es $\phi-$demostrable.
\end{defi}
En general, una prueba matemática es una secuencia de afirmaciones, llamadas pasos de la prueba, que conducen a la conclusión deseada. Con esta definición, una demostración de una fórmula $b$ en un conjunto de reglas $\phi$ es una secuencia de fórmulas $a_1, a_2, \ldots, a_n$ tal que $a_n=b$ y para cada $a_i$, si $i \leq n$, entonces existe un subconjunto finito $X$ de ${a_1, a_2, \ldots, a_{i-1}}$ tal que $\phi(X)=a_i$. Es decir, cada fórmula en la secuencia es obtenida a partir de las fórmulas previas mediante la aplicación de alguna regla de $\phi$ cuyas premisas son un subconjunto finito de las fórmulas previas.

\begin{propo}.\\
Para $\phi$ un conjunto de reglas finito:
$$I(\phi) = \{b \quad: b\,\, tiene\,\, una\,\, prueba\}$$
\end{propo}

\subsection{La parte bien fundamentada de una relación}
Sea $A$ un conjunto, $<$ una relación de orden parcial en $A$, definimos \textbf{la parte bien fundamentada de $<$} como 

$$W(<) = \{a\in A\,\, : \forall \{a_i\}\subseteq A (a_0 = a, a_{i+1}<a_i\rightarrow |\{a_i\}|<\infty)\}$$

Es decir el conjunto de los elementos sin secuencias infinitas de descendientes.\\ 
Decimos que $<$ es bien fundamentada, si $w(<) = A$

\begin{propo}\textbf{$w(<)$ definido inductivamente}\\
    $w(<)$ puede ser definido inductivamente del conjunto de reglas definido por:
    \begin{itemize}
        \item $((<a),a) \in\phi_<$ para $a\in A$, donde $(<a) = \{x\in A: x<a\}$    
    \end{itemize} 
\end{propo}

\begin{propo}\textbf{Otra interpretación de un conjunto definido inductivamente}
    $$W(<) = I(\phi_<) = \{b \quad: b\,\, tiene\,\, una\,\, prueba\}$$
    Es decir El conjunto inductivamente definido por el conjunto de reglas definido por una relación de orden parcial es la parte bien fundamentada de $<$ y a su vez el conjunto de los elementos $\varphi_<$-demostrables.
\end{propo}

\begin{defi}\textbf{Conjuntos de reglas deterministas}\\
    El conjunto de reglas $\phi$ es llamado \textbf{determinista} si $\phi:X_1\rightarrow x$ y $\phi:X_2\rightarrow x$ implica que $X_1=X_2$
\end{defi}
\begin{ejemplo}
Veamos un par de ejemplos interesantes:
    \begin{itemize}
        \item $\phi_<$ Siempre es determinista.\\
        $\phi_< $ es determinista porque para dos subconjuntos $X_1$ y $X_2$ tales que $\phi_< : X_1\rightarrow x$ y $\phi_< : X_2\rightarrow x$, si $X_1 \neq X_2$ entonces existe un elemento $a \in X_1\Delta X_2$ que es el mínimo elemento en $X_1\Delta X_2$, donde $\Delta$ denota la diferencia simétrica entre conjuntos. Entonces, sin pérdida de generalidad, podemos suponer que $a\in X_1$ y $a\notin X_2$, lo que significa que para cualquier $b \in X_2$ se tiene $a<b$, y por lo tanto $\phi_<(X_2) \subseteq (<a)$, lo que contradice el hecho de que $\phi_<(X_2) = x$. Por lo tanto, $X_1 = X_2$ y $\phi_<$ es determinista.
        \item El conjunto de reglas que define los terminos y formulas de la lógica de primer orden no es determinista.\\
        El conjunto de reglas para la lógica de primer orden consiste en:
        \begin{enumerate}
            \item El conjunto $\mathcal{V}$ de variables.
            \item Un conjunto $\mathcal{C}$ de constantes.
            \item Para cada $n$ un conjunto $\mathcal{F}_n$ de símbolos de función de aridad $n$.
            \item Para cada $n$ un conjunto $\mathcal{R}_n$ de símbolos de relación de aridad $n$.
            \item Un conjunto $\mathcal{L}$ de símbolos lógicos, que incluye al menos a $\neg$, $\vee$, $\wedge$, $\rightarrow$, $\leftrightarrow$, $\forall$, y $\exists$.
        \end{enumerate}
        A partir de estos conjuntos, podemos construir los términos y fórmulas de la lógica de primer orden mediante las siguientes reglas:
        \begin{itemize}
            \item Todo elemento de $\mathcal{V} \cup \mathcal{C}$ es un término.
            \item Si $f \in \mathcal{F}_n$ y $t_1,\ldots,t_n$ son términos, entonces $f(t_1,\ldots,t_n)$ es un término.
            \item Si $R \in \mathcal{R}_n$ y $t_1,\ldots,t_n$ son términos, entonces $R(t_1,\ldots,t_n)$ es una fórmula atómica.
            \item Si $A$ y $B$ son fórmulas, entonces $\neg A$, $(A \vee B)$, $(A \wedge B)$, $(A \rightarrow B)$, y $(A \leftrightarrow B)$ son fórmulas.
            \item Si $A$ es una fórmula y $x \in \mathcal{V}$, entonces $\forall x,A$ y $\exists x,A$ son fórmulas.
        \end{itemize}
        Este conjunto de reglas no es determinista, ya que, por ejemplo, podemos tener dos símbolos de función distintos que tengan el mismo nombre pero aridades diferentes, lo que llevaría a una ambigüedad en la interpretación de las fórmulas que los usan.
    \end{itemize}
\end{ejemplo}

Ahora bien, sea $\phi$ determinista y $A$ un conjunto de conclusiones de reglas en $\phi$. para $x,y\in A$ sea $x<y$ si $\phi: X\rightarrow y$ para algun conjunto $X$ tal que $x\in X$ y $X\subseteq A$. Entonces $\phi_<$ es el conjunto de reglas $X\rightarrow x$ en $\phi$ tal que $X\subseteq A$. Además tenemos:
\begin{propo}
    Si $\phi$ es determinista:
    $$I(\phi)=I(\phi_<)$$
\end{propo}
La proposición establece que si $\phi$ es determinista, entonces el conjunto de fórmulas demostrables mediante $\phi$ es igual al conjunto de fórmulas demostrables mediante $\phi_<$, es decir, $I(\phi) = I(\phi_<)$.

\subsection{Definiciones inductivas como operadores}
Sea $\varphi:Pow(A)\rightarrow Pow(A)$, donde $Pow(A)=\{X:X\subseteq A\}$. El operador $\phi$ es \textbf{monotono} si $X\subseteq Y \subseteq A$ implica que $\varphi(X)\subseteq \varphi(Y) \subseteq A$.\\ 
Dado $\varphi:Pow(A)\rightarrow Pow(A)$ sea $\phi_{\varphi}$ el conjunto de reglas $X\rightarrow x$ tal que $X\subseteq A$ y $x \in \varphi(X)$. Para $\phi$ monotona, $X \subseteq A$ es $\phi_{\varphi}$-cerrado justo en el caso $\varphi(X) \subseteq X$. Asi $I(\phi_{\varphi})=\bigcap\{X \subseteq A: \varphi(X) \subseteq X\}$. Además, es natural extender la terminología concerniente a definiciones inductivas a operadores monotonos $\varphi$ y escribimos $I(\varphi)=\bigcap\{X \subseteq A: \varphi(X) \subseteq X\}$ y lo llamaremos el \textbf{conjunto intuctivamente definido por $\varphi$.} Todas las definiciones inductivas pueden ser obtenidas usando operadores monotonos. Para $\phi$ un conjunto de reglas en $A$ ($X\cup\{x\}\subseteq A$ cuando $\phi:X\rightarrow x$) Podemos definir un operador monotono mediante $\varphi:Pow(A)\rightarrow Pow(A)$ por:
$$\varphi(Y)=\{x\in A:\exists X\subseteq Y (\phi:X\rightarrow x)\}$$
Entonces $Y\subseteq A$ es $\phi$-cerrado cuando $\phi(Y)\subseteq Y$, así $I(\varphi)=I(\phi)$.\\

Sea $\varphi : Pow(A) \rightarrow Pow(A)$ un operador monotono, entonces se define $\varphi^\lambda\subseteq A$
$$\varphi^\lambda = \bigcup_{\mu<\lambda}\varphi^{\mu}\cup(\bigcup_{\mu<\lambda}\varphi^{\mu}) $$
Entonces $\varphi^{\infty}=\bigcup_{\lambda}\varphi^{\lambda}$
\begin{propo}
    Para $\phi:Pow(A)\rightarrow Pow(A)$ monotono:
    \begin{enumerate}
        \item $I(\varphi)=\varphi^{\infty}$
        \item $I(\varphi)$ es el punto fijo más pequeño de $\varphi$
    \end{enumerate}

\end{propo}
\subsection*{Concepto de prueba para inducción monotona}
Un cardinal regular es un cardinal $\kappa$ tal que para cualquier colección de conjuntos $\{A_i\}_{i\in I}$, si $\vert A_i \vert < \kappa$ para todo $i\in I$, entonces $\left\vert \bigcup_{i\in I} A_i \right\vert < \kappa$.

\begin{defi}
    Sea $\varphi$ un operador monotono en $A$. Una secuencia transfinita $\{a_{\mu}\}_{\mu\leq\lambda}$ es una $\varphi$-prueba de $b$ con longitud $\lambda$ si 
    \begin{enumerate}
        \item $a_{\lambda} =b$
        \item $a_{\nu}\in\varphi\{a_{\mu}\,\,:\mu<\nu\}$ para todo $\nu\leq\lambda$
    \end{enumerate}
\end{defi}
La definición de $\varphi$-prueba establece una condición para demostrar que un elemento $b\in A$ pertenece a $\varphi^\lambda$ para alguna $\lambda$. Es decir, una $\varphi$-prueba de $b$ es una secuencia de elementos en $A$ que cumple ciertas condiciones.

La primera condición de la definición es que el último elemento de la secuencia es precisamente $b$. La segunda condición es que cada elemento en la secuencia puede ser obtenido aplicando el operador $\varphi$ a la colección de elementos previos de la secuencia.\\

Por otra parte definimos $\varphi^{<\lambda}$ recursivamente por:
$$\varphi^{<0} = \emptyset, \,\,\,\,\,\, \varphi^{<\lambda + 1} = \varphi^{<\lambda} \cup \varphi(\varphi^{<\lambda})$$
\begin{center}
    $\varphi^{<\lambda} = \bigcup_{\mu<\lambda} \varphi^{<\mu}$ para un limite $\lambda$
\end{center}

\begin{propo}
    \begin{enumerate}
        \item Para cualquier cardinal regular $\kappa$\\
        $\varphi^{<\kappa}=\{a\in A \,\,\,; \exists_{\{a_{mu}\}_{\nu<\kappa}}(a_{\kappa} =b \land \forall_{\nu\leq\kappa}a_{\nu}\in\varphi\{a_{\mu}\,\,:\mu<\nu\}\}$ Es decir $a$ tiene una $\varphi$-prueba de longitud $\kappa$.
        \item $I(\varphi)=\{\{a\in A \,\,\,; \exists_{\kappa}\exists_{\{a_{mu}\}_{\nu<\kappa}}(a_{\kappa} = b \land \forall_{\nu\leq\kappa}a_{\nu}\in\varphi\{a_{\mu}\,\,:\mu<\nu\}\}\}$ es decir que $I(\varphi)$ es el conjunto de formulas $\varphi$-probables.
    \end{enumerate}
\end{propo}
La proposición establece que el conjunto $I(\varphi)$ de puntos fijos de $\varphi$ es precisamente el conjunto de elementos en $A$ que tienen una $\varphi$-prueba de longitud $\kappa$ para cualquier cardinal regular $\kappa$. En otras palabras, un elemento $a\in A$ pertenece a $I(\varphi)$ si y solo si existe una $\varphi$-prueba infinita que comienza con $a$ y cumple con las condiciones de la definición. De esta manera, $I(\varphi)$ se puede entender como el conjunto de elementos en $A$ que son $\varphi$-probables bajo el operador, ya que tienen una prueba infinita que los justifica.

\begin{defi}\textbf{Un arbol bien fundamentado}\\
    Un arbol bien fundamentado $T$ es un conjunto de secuencias finitas de longitud mayor que $0$ tal que
    \begin{enumerate}
        \item Hay exactamente una sequencia de longitud $1$ en $T$, es llamada la raíz $(a_{T})$ del arbol.
        \item si $(a_1,\ldots,a_{n+1})\in T$, entonces $(a_1,\ldots,a_{n})\in T$
        \item $T$ es bien fundamentado en el sentido de que no hay sequencias infinitas $a_1,\ldots,a_n,\ldots$ tal que $(a_1,\ldots,a_{n})\in T$ para cada $n>0$.\\
        Alternativamente $<_T$ es bien fundamentada, cuando:
        $$(a_1,\ldots,a_{m}) = (b_1,\ldots,b_{n}) \,\,\,\ leftrightarrow \,\,\,\ n = m+1$$
        y $a_i = b_i$ para $i\in\{1,\ldots, m\}$
    \end{enumerate}
\end{defi}
Un árbol bien fundamentado es un conjunto de secuencias finitas que cumple ciertas condiciones. La primera condición es que hay una única secuencia de longitud 1, que es la raíz del árbol. La segunda condición es que si una secuencia $(a_1,\ldots,a_{n+1})$ está en el árbol, entonces su prefijo $(a_1,\ldots,a_n)$ también debe estar en el árbol. La tercera condición es que el árbol es "bien fundamentado", lo que significa que no hay secuencias infinitas en el árbol.

\begin{defi}
    Si $\phi$ es un conjunto de regla, un arbol $T$ es una $\phi$-prueba de un $a = a_T$ y $\phi: T_{(a_1,\ldots,a_{n})}\rightarrow a_n$, siempre que $(a_1,\ldots,a_{n})\in T$, donde:
    $$T_{(a_1,\ldots,a_{n})} = \{a \,\, : (a_1,\ldots,a_{n},a) \in T\}$$
\end{defi}
\begin{propo}
    \begin{enumerate}
        \item $I(\phi) = \{a\,\,:\exists_{T}(T\rightarrow a)\}$ el conjunto de elementos tal que tienen un arbol que lo $\phi$-prueban.
        \item si $\phi$ es un conjunto de reglas en $A$ con su correspondiente operador monotono $\varphi:Pow(A)\rightarrow Pow(A)$, entonces para todo ordinal $\lambda$,
        $$\varphi^{\lambda} = \{a\,\,:\exists_{T}(|T|\leq\lambda\,\,\land\,\,\,T\rightarrow a)\}$$
    \end{enumerate}
\end{propo}

\subsection{Puntos fijos en un lattice completo}
\begin{defi}\textbf{Poset}\\
    Un conjunto parcialmente ordenado o poset, es un conjunto no vacío equipado con una relación de orden parcial, es decir \textbf{reflexiva, antisimetrica y transitiva}, usualmente lo denotamos como $(A,<)$, donde $A$ es el conjunto y $<$ es la relación de orden parcial
\end{defi}
A continuación algunos ejemplos:
\begin{ejemplo}\textbf{Conjuintos ordenados}
    \begin{enumerate} 
        \item El conjunto de los números enteros $\mathbf{Z}$, junto con la relación de orden parcial $<$. Es decir, para $a, b \in \mathbf{Z}$, $a < b$ si y solo si $a$ es menor o igual que $b$.
        \item El conjunto de los subconjuntos de un conjunto dado $X$, junto con la relación de orden parcial "$\subset$". Es decir, para $A, B \subset X$, $A \subset B$ si y solo si todos los elementos de $A$ también están en $B$.
        \item El conjunto de los productos cartesianos $A \times B$ de dos conjuntos $A$ y $B$, junto con la relación de orden parcial $<_{\times}$ llamado \textbf{orden lexicográfico}. Es decir, para $(a_1,b_1),(a_2,b_2) \in A \times B$, $(a_1,b_1) <_{\times} (a_2,b_2)$ si y solo si $a_1 < a_2$ o ($a_1 = a_2$ y $b_1 < b_2$).
    \end{enumerate}
\end{ejemplo}
\begin{ejemplo}\textbf{Preorden}\\
    Un conjunto preordenado es un conjunto no vacío equipado con una relación \textbf{reflexiva y transitiva}. Si $L$ es este preorden definimos la siguiente relación de equivalencia como.
    $$x \sim y \quad si\,\, y \,\, solo \,\, si \quad x \geq y \,\, \land \,\, y \geq x$$ 
    definimos $A = L/\sim$, con el orden inducido por esta relación de equivalencia.
\end{ejemplo}

\begin{defi}\textbf{Infimo y supremo}\\
    Sea $\mathbf{R}$ un poset, definimos:
\begin{itemize}
    \item \textbf{infimo (met):} Dado un conjunto $A \subseteq \mathbf{R}$, $m \in \mathbf{R}$ se llama infimo de $A$ si se cumple que:
    \begin{enumerate}
        \item $m$ es cota inferior de $A$, es decir, $m \leq a$ para todo $a \in A$.
        \item Cualquier otra cota inferior de $A$ es mayor o igual que $m$, es decir, si $m' \leq a$ para todo $a \in A$, entonces $m' \geq m$.
    \end{enumerate}
    El infimo de $A$ se denota como $\inf(A)=met(A)$.
    \item \textbf{supremo (join):}  Dado un conjunto $A \subseteq \mathbf{R}$, $M \in \mathbf{R}$ se llama supremo de $A$ si se cumple que:
    \begin{enumerate}
        \item $M$ es cota superior de $A$, es decir, $M \geq a$ para todo $a \in A$.
        \item Cualquier otra cota superior de $A$ es menor o igual que $M$, es decir, si $M' \geq a$ para todo $a \in A$, entonces $M' \leq M$.
    \end{enumerate}
    El supremo de $A$ se denota como $\sup(A)=join(A)$.
\end{itemize}    
\end{defi}

\begin{ejemplo}\textbf{Dual de un poset}\\
    Dado un poset $(A,<)$, su dual o inverso $(A^{op},>)$ es el poset que se obtiene al invertir la relación de orden, es decir, $a>b$ en el dual si y solo si $b>a$ en el poset original. Formalmente, $(A{op},>)$ es un poset con $A$ como conjunto y la relación de orden $>$ definida como:
    $$a > b \,\,en\,\,A^{op} \quad si\,\, y\,\, solo\,\, si \quad b < a\,\,en\,\,A \quad para\,\, todo \quad a,b \in A$$
    El dual de un poset conserva las propiedades de reflexividad, antisimetría y transitividad.

    si $(A, <)$ es un poset y $B \subseteq A$ tiene un ínfimo $m$ en $(A, <)$, entonces $m$ es el supremo de $B$ en $(A^{op}, >)$. Por otro lado, si $B$ tiene un supremo $M$ en $(A, <)$, entonces $M$ es el ínfimo de $B$ en $(A^{op}, >)$.
\end{ejemplo}

\begin{defi}
    Dado un poset $(L, \leq)$ y una endofunción $F: L \rightarrow L$, definimos:
    \begin{itemize}
        \item $F$ es \textbf{monótono} si preserva el orden parcial, es decir, para todo $x, y \in L$, si $x \leq y$, entonces $f(x) \leq f(y)$.
        \item Los \textbf{puntos pre-fijos} de una endofunción $f: L \rightarrow L$ son los elementos $x \in L$ tales que $f(x) \leq x$. De manera similar, los \textbf{puntos post-fijos} son los elementos $x \in L$ tales que $x \leq f(x)$.\\ 
        Es decir, un punto pre-fijo es un elemento que el endofunción no puede llevar más allá de sí mismo, mientras que un punto post-fijo es un elemento que el endofunción no puede llevar por debajo de sí mismo.
        \item Los \textbf{puntos fijos} de una endofunción $f: L \rightarrow L$ son los elementos $x \in L$ tales que $f(x) = x$. Es decir, son los elementos que el endofunción deja inalterados.
    \end{itemize} 
\end{defi}

\begin{ejemplo}\textbf{El algoritmo de la división como una endofunción}\\
    Consideremos el conjunto $A = \mathbf{N}^2$ con el orden lexicográfico definido como $(a,b) < (c,d)$ si $a < c$ o si $a = c$ y $b < d$. Sea $f: A \rightarrow A$ el endofunción definido por el algoritmo de la división, es decir, $f(x,y) = (m,r)$ si $0 < y$, $x = ym + r$, donde $0 \leq r < y$, y si $y=0$, $f(x,0) = (0,x)$ 

    \begin{itemize}
        \item Un punto prefijo de $f$ es un elemento $(x,y) \in A$ tal que $f(x,y) \leq (x,y)$ en el orden lexicográfico. En este caso, podemos ver que si $(x,y)\leq(m,r)$ si $x < m$ o si $x = m$ y $y \leq r$, vemos que en todo momento $x \geq m$, por lo que $x = m$ y $y \leq r$, por lo tanto $y = 0$, y por tanto $x = 0$, y vemos que es también el unico punto fijo
        \item Un punto postfijo de $f$ es un elemento $(x,y) \in A$ tal que $(x,y) \leq f(x,y)$ en el orden lexicográfico. En este caso, podemos observar que $(x,y)\geq(m,r)$ es un punto postfijo si $m < x$ o si $x = m$ y $r \leq y$, entonces, si $x \geq y$ o $y = 0$
        \item Un punto fijo de $f$ es un elemento $(x,y) \in A$ tal que $f(x,y) = (x,y)$ en el orden lexicográfico. En este caso, podemos observar que $(0,0)$ es el único punto fijo de $f$, ya que $f(0,0) = (0,0)$ y no hay otro elemento $(x,y)$ que sea igual a $(0,0)$ en el orden lexicográfico.
    \end{itemize}
    Es facil ver que no es un monomorfismo.
\end{ejemplo}

\begin{defi}\textbf{enrejado (lattice) completo}\\
    Un lattice completo es un conjunto parcialmente ordenado $(L,\leq)$ en el que todos los subconjunto no vacío tiene supremo.
\end{defi}
Note que basta con exigir la existencia del supremo en todo conjunto para tener la existencia del infimo en todo conjunto. Esto por la definición. Es decir en un lattice completo todo subconjunto no vacío posee infimo. También tenemos que los lattices completos son acotados, es decir poseen maximo y minimo.

\begin{teo}\textbf{Teorema del punto fijo}\\
    En un lattice completo, una endofunción tiene un lattice completo de puntos fijos. En particlar el punto fijo más pequeño de la funcíon es el infimo de los puntos pre-fijos, el más grande es el supremo de los puntos postfijos. Más aún por la condición de sobreyectividad estos son el minimo y el máximo del lattice. 

\subsection{Conjuntos definidos inductiva y coinductivamente}

\end{teo}
\newpage
\begin{thebibliography}{9}
    \bibitem{Aczel}
    Aczel, P. (1997). 
    An Introduction to Inductive Definitions
\end{thebibliography}
\end{document} 