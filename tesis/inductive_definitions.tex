\documentclass[executivepaper]{article}

\usepackage{graphicx}
\usepackage[utf8]{inputenc}
\usepackage[T1]{fontenc}
\usepackage[spanish]{babel} % Establece el idioma español
\usepackage{csquotes} % Carga el paquete csquotes
\usepackage{graphicx} % Required for inserting images
\usepackage{listings}
\usepackage{xcolor}
\usepackage{hyperref}
\usepackage[left=1.50cm, right=1.50cm]{geometry}
\usepackage{tikz}
\usetikzlibrary{shapes,arrows}
\usetikzlibrary{positioning}
\setlength{\parindent}{0.5in}
\usepackage{setspace}
\usepackage{amssymb}
\doublespacing

\lstset{
    inputencoding=utf8,
    language=Java,
    basicstyle=\ttfamily,
    columns=fullflexible
}

% Define colores para el código
\definecolor{codegreen}{rgb}{0,0.6,0}
\definecolor{codegray}{rgb}{0.5,0.5,0.5}
\definecolor{codepurple}{rgb}{0.58,0,0.82}
\definecolor{backcolour}{rgb}{0.95,0.95,0.92}

% Configuración de lstlisting
\lstdefinestyle{mystyle}{
    backgroundcolor=\color{backcolour},   
    commentstyle=\color{codegreen},
    keywordstyle=\color{magenta},
    numberstyle=\tiny\color{codegray},
    stringstyle=\color{codepurple},
    basicstyle=\ttfamily\footnotesize,
    breakatwhitespace=false,         
    breaklines=true,                 
    captionpos=b,                    
    keepspaces=true,                 
    numbers=left,                    
    numbersep=5pt,                  
    showspaces=false,                
    showstringspaces=false,
    showtabs=false,                  
    tabsize=2
}

% Configuración del paquete hyperref
\hypersetup{
    colorlinks=true,
    linkcolor=black,
    filecolor=magenta,      
    urlcolor=gray,
}

\lstset{style=mystyle}

\renewcommand{\baselinestretch}{1.5}

\newtheorem{propo}{Proposición}[section]
\newtheorem{lema}[propo]{Lema}
\newtheorem{teo}[propo]{Teorema}
\newtheorem{coro}[propo]{Corolario}
\newtheorem{defi}[propo]{Definición}
\newtheorem{obs}[propo]{Observación}
\newtheorem{ejemplo}[propo]{Ejemplo}

\newcommand{\Al}{(\mathcal{A},\mathds{F},\odot)}
\newcommand{\A}{\mathcal{A}}
\newcommand{\B}{\mathcal{B}}
\newcommand{\D}{\mathcal{D}}
\newcommand{\C}{\mathcal{C}}
\newcommand{\I}{\mathcal{I}}
\newcommand{\J}{\mathcal{J}}
\newcommand{\R}{\mathds{R}}
\newcommand{\N}{\mathbb{N}}
\newcommand{\Z}{\mathbb{Z}}
\newcommand{\fu}{f:D\longrightarrow \mathds{R}}
\newcommand{\fun}{f:[a,b]\longrightarrow \mathds{R}}
\newcommand{\E}{\mathcal{E}}
\newcommand{\F}{\mathds{F}}
\newcommand{\op}{``}
\newcommand{\cl}{''}
\newcommand{\po}{^}
\newcommand{\Q}{\matbbb{Q}}

\title{Definiciones inductivas}
\author{Kevin Cárdenas}

\begin{document}
\begin{titlepage}
    \begin{center}
        {\Huge \textbf{Definiciones Inductivas y Coinductivas}}
        \\[8cm]
        
        \large\emph{Autor:}\\
        Kevin Cárdenas.
        \\[6cm]
        \large\emph{Asesor:}
        Juan Carlos Agudelo.\\
        Trabajo de grado en modalidad de monografía para el titulo como matemático.\\
        2023
    \end{center}
\end{titlepage}

\newpage
\tableofcontents
\newpage

\section{Definiciones inductivas}

Se conoce como inducción matemática al método de demostración que sirve para demostrar afirmaciones que son verdaderas para todos los números naturales. Se basa en la idea de que si se demuestra que una afirmación es verdadera para el número natural \emph{base} ($\phi(0)$), y luego se muestra que si es verdadera para cualquier número natural $n$, entonces también lo es para el siguiente número natural $n+1$, se concluye que la afirmación es verdadera para todos los números naturales mayores a ese punto base.

Los conjuntos definidos inductivamente se presentan, de manera informal, a partir de una definición inicial, para unos puntos base en el conjunto, y unas reglas que definen como añadir elementos a este. El conjunto en cuestión sería el más pequeño que cumple estas reglas y contienen estos puntos iniciales. Se presenta el ejemplo por excelencia de un conjunto inductivamente definido.
\begin{ejemplo} \textbf{El conjunto de los números Naturales $\N$}\\
El conjunto de los números naturales es el \emph{menor conjunto} tal que:
\begin{enumerate}
    \item $0 \in \N$
    \item Si $n \in \N$, entonces $s(n )\in \N$
\end{enumerate}
Los elementos en $\N$ son solo los producidos por las reglas 1) y 2).

Resaltando que la definición se basa solo en generar un conjunto con cierta estructura, es posible encontrar diferentes formas de definir cada conjunto. Por ejemplo, se puede definir explicitamente la función sucesor como $S(n) = n + 1$.
\end{ejemplo}

\begin{ejemplo}\textbf{El conjunto de formulas de la lógica proposicional clásica (CPL)}\\
    Dado un conjunto $V = \{p_1, p_2,...\}$ de \emph{variables proposicionales}, el conjunto $Form(CPL)$ de \emph{formulas de la lógica proposicional clásica} es el menor conjunto tal que:
    \begin{enumerate}
        \item $V \subseteq Form(CPL)$
        \item Si $A, B \in Form(CPL)$, entonces $\neg A, (A \lor B), (A \land B), (A \rightarrow B), (A \leftrightarrow B) \in Form(CPL)$
    \end{enumerate}
    Se hace enfasis en \emph{Cómo} se generan los elementos del conjunto y cuáles son los elementos iniciales.
\end{ejemplo}

Otro ejemplo que será de útilidad es el conjunto de listas sobe un conjunto dado.
\begin{ejemplo}\textbf{El conjunto de listas finitas de un conjunto A}\\
    Dado un conjunto $A$, el conjunto $Finlist(A)$
    \begin{enumerate}
        \item $nill \in Finlist(A)$
        \item si $S \in Finlist(A)$ y $a \in A$ , entonces $\langle a \rangle\cdot S\in Finlist(A)$
    \end{enumerate}
El operador $\cdot$ funciona como un metodo para concatenar un elemento en una lista ya existente, entonces todos los terminos en $Finlist(A)$ son de la forma $\rangle a_n\langle \cdot\rangle a_{n-1} \langle\cdot\rangle a_{n-2}langle\cdot\ldots\cdot\rangle a_{0}langle\cdot nill$.
\end{ejemplo}

\subsection{Conjuntos Inductivamente Definidos a partir de un conjunto de reglas}
\begin{defi}
\textbf{Reglas, Conjunto $\phi-$cerrado y Conjunto inductivamente definido}
\begin{enumerate}
\item Una regla es un par $(A,x)$, donde $A$ es un conjunto, llamado \textbf{el conjunto de premisas}, y $x$ es la conclusión.
\item Si $\phi$ es un conjunto de reglas, entonces un conjunto $A$ es $\phi - \textbf{cerrado}$ si para toda regla $(X,x) \in \phi$ si $X \subseteq A$ entonces $x \in A$.
\item Dado $\phi$ un conjunto de reglas, definimos $I(\phi)$ como el conjunto inductivamente definido por $\phi$, dado por:
$$I(\phi) = \bigcap \{A : A \,\,\phi{-cerrado}\}$$
\end{enumerate}
\end{defi}

\begin{ejemplo}\textbf{$\N$ Como conjunto inductivamente definido a partir de un conjunto de reglas}\\
Definimos $\N = I(\phi)$, donde $\phi$ es el conjunto que contienen las siguientes dos reglas:
\begin{enumerate}
\item $(\emptyset,0)$
\item $(\{n\},s(n))$
\end{enumerate}
Note que, claramente $\N$ es $\phi$-cerrado y además si $A$ es $\phi$-cerrado, entonces $A \subseteq \N$
\end{ejemplo}

\begin{ejemplo}\textbf{$Form(CPL)$ Como conjunto inductivamente definido a partir de un conjunto de reglas}\\
Sea $\phi$ el conjunto con las siguientes reglas:
\begin{enumerate}
    \item $(\emptyset, p_i)$ para cada $p_i \in V$
    \item $(\{A\}, \neg A)$
    \item $(\{A, B\}, A \lor B)$
    \item $(\{A, B\}, A \land B)$
    \item $(\{A, B\}, A \rightarrow B)$
    \item $(\{A, B\}, A \leftrightarrow B)$
\end{enumerate}
Obteniendo que $Form(CPL) = I(\phi)$.
\end{ejemplo}

\begin{ejemplo}\textbf{$Finlist(A)$ Como conjunto inductivamente definido a partir de un conjunto de reglas}\\
Dado un conjunto $A$, y $\phi$ el conjunto que contiene las siguientes reglas:
\begin{enumerate}
    \item $(\emptyset, nill)$
    \item $(\{S\},\langle a \rangle\cdot S)$ 
\end{enumerate}
Se tiene que $Finlist(A) = I(\phi)$
\end{ejemplo}

\begin{defi}
\textbf{$\phi-$prueba}\\
$a_1, a_2, \ldots, a_n$ es una $\phi-$prueba de $b$ si
\begin{enumerate}
\item $a_n = b$
\item si $m \leq n$, entonces existe $X\subseteq \{a_i\}_{i\geq m}$ tal que $\phi : X \rightarrow a_m$
\end{enumerate}
Si $x$ tiene una $\phi-$prueba, entonces decimos que $x$ es $\phi$-demostrable.
\end{defi}

\begin{propo}.\\
Para $\phi$ un conjunto de reglas finito:
$$I(\phi) = \{b \,\,:b\,\, {tiene}\,\, {una}\,\, \phi {-prueba} \}$$
De la definición es claro que, primero $\{b \,\,: b\,\, tiene\,\, una\,\, \phi{-prueba}\}$ es $\phi$-cerrado, y además para cada $x \in I(\phi)$ $x$ es $\phi$-demostrable. (La demostración se encuentra en la pagina $743$ del libro de Aczel Introduction to Inductive Definitions.)
\end{propo}

\subsection{Conjuntos inductivamente definidos como puntos fijos en un reticulo completo}
\begin{defi}\textbf{Conjunto parcialmente ordenado}\\
    Un conjunto parcialmente ordenado (CPO), es un conjunto no vacío equipado con una relación de orden parcial, es decir \textbf{reflexiva, antisimetrica y transitiva}, lo denotamos como $(A,\geq)$, donde $A$ es el conjunto y $\geq$ es la relación de orden parcial
\end{defi}
A continuación algunos ejemplos:
\begin{ejemplo}\textbf{Conjuintos ordenados}
    \begin{enumerate} 
        \item El conjunto de los números enteros $\Z$, junto con la relación de orden usual $\geq$.
        \item El conjunto de los subconjuntos de un conjunto dado $X$, junto con la relación de orden parcial "$\subset$". Es decir, para $A, B \subset X$, $A \subset B$ si y solo si todos los elementos de $A$ también están en $B$.
        \item El conjunto de los productos cartesianos $A \times B$ de dos CPOs $(A,\geq_A)$ y $(B,\geq_B)$, junto con la relación de orden parcial $\geq_{\times}$ llamado \textbf{orden lexicográfico}. Es decir, para $(a_1,b_1),(a_2,b_2) \in A \times B$, $(a_1,b_1) \geq_{\times} (a_2,b_2)$ si y solo si $a_1 \geq_A a_2$ o ($a_1 = a_2$ y $b_1 \geq_B b_2$).
    \end{enumerate}
\end{ejemplo}
\begin{ejemplo}\textbf{Preorden}\\
    Un conjunto preordenado es un conjunto no vacío equipado con una relación \textbf{reflexiva y transitiva}. Si $L$ es este preorden definimos la siguiente relación de equivalencia como.
    $$x \sim y \quad si\,\, y \,\, solo \,\, si \quad x \geq y \,\, \land \,\, y \geq x$$ 
    definimos $A = L/\sim$, con el orden inducido por esta relación de equivalencia.
\end{ejemplo}

\begin{defi}\textbf{Infimo y supremo}\\
    Sea $R$ un CPO, definimos:
\begin{itemize}
    \item \textbf{infimo (meet):} Dado un conjunto $A \subseteq R$, $m \in R$ se llama infimo de $A$ si se cumple que:
    \begin{enumerate}
        \item $m$ es cota inferior de $A$, es decir, $m \leq a$ para todo $a \in A$.
        \item Cualquier otra cota inferior de $A$ es mayor o igual que $m$, es decir, si $m' \leq a$ para todo $a \in A$, entonces $m' \geq m$.
    \end{enumerate}
    El infimo de $A$ se denota como $\inf(A)=met(A)$.
    \item \textbf{supremo (join):}  Dado un conjunto $A \subseteq R$, $M \in R$ se llama supremo de $A$ si se cumple que:
    \begin{enumerate}
        \item $M$ es cota superior de $A$, es decir, $M \geq a$ para todo $a \in A$.
        \item Cualquier otra cota superior de $A$ es menor o igual que $M$, es decir, si $M' \geq a$ para todo $a \in A$, entonces $M' \leq M$.
    \end{enumerate}
    El supremo de $A$ se denota como $\sup(A)=join(A)$.
\end{itemize}    
\end{defi}

\begin{ejemplo}\textbf{Dual de un CPO}\\
    Dado un CPO $(A,\geq)$, su dual o inverso $(A^{op},>)$ es el CPO que se obtiene al invertir la relación de orden, es decir, $a>b$ en el dual si y solo si $b>a$ en el CPO original. Formalmente, $(A{op},>)$ es un CPO con $A$ como conjunto y la relación de orden $>$ definida como:
    $$a > b \,\,en\,\,A^{op} \quad si\,\, y\,\, solo\,\, si \quad b \geq a\,\,en\,\,A \quad para\,\, todo \quad a,b \in A$$
    El dual de un CPO conserva las propiedades de reflexividad, antisimetría y transitividad.

    si $(A, \geq)$ es un CPO y $B \subseteq A$ tiene un ínfimo $m$ en $(A, \geq)$, entonces $m$ es el supremo de $B$ en $(A^{op}, >)$. Por otro lado, si $B$ tiene un supremo $M$ en $(A, \geq)$, entonces $M$ es el ínfimo de $B$ en $(A^{op}, >)$.
\end{ejemplo}

\begin{defi}
    Dado un CPO $(L, \leq)$ y una endofunción $F: L \rightarrow L$, definimos:
    \begin{itemize}
        \item $F$ es \textbf{monótono} si preserva el orden parcial, es decir, para todo $x, y \in L$, si $x \leq y$, entonces $f(x) \leq f(y)$.
        \item Los \textbf{puntos pre-fijos} de una endofunción $f: L \rightarrow L$ son los elementos $x \in L$ tales que $f(x) \leq x$. De manera similar, los \textbf{puntos post-fijos} son los elementos $x \in L$ tales que $x \leq f(x)$.\\ 
        Es decir, un punto pre-fijo es un elemento que el endofunción no puede llevar más allá de sí mismo, mientras que un punto post-fijo es un elemento que el endofunción no puede llevar por debajo de sí mismo.
        \item Los \textbf{puntos fijos} de una endofunción $f: L \rightarrow L$ son los elementos $x \in L$ tales que $f(x) = x$. Es decir, son los elementos que el endofunción deja inalterados.
    \end{itemize} 
\end{defi}

\begin{defi}\textbf{Reticulo completo}\\
    Un reticulo completo es un conjunto parcialmente ordenado $(L,\leq)$ en el que todos los subconjunto tiene supremo.
\end{defi}
Note que basta con exigir la existencia del supremo en todo conjunto para tener la existencia del infimo en todo conjunto. Esto por la definición. Es decir en un reticulo completo todo subconjunto no vacío posee infimo. También tenemos que los reticulos completos son acotados, es decir poseen maximo y minimo.

\begin{teo}\textbf{Teorema del punto fijo}\\
    En un reticulo completo $L$, una endofunción monotona $F:L\rightarrow L$ tiene un reticulo completo de puntos fijos. En particlar el punto fijo más pequeño de la funcíon es el infimo de los puntos pre-fijos, el más grande es el supremo de los puntos postfijos. Más aún, si se da la sobreyectividad estos son el minimo y el máximo del reticulo de puntos fijos. 
\end{teo}
La demostración de este teorema se encuentra, como ejercicio, en el libro Introduction to bisimulation and coinduccion.
\subsection*{Conjuntos definidos inductiva y coinductivamente}

\begin{defi}\textbf{Conjuntos definidos inductiva y coinductivamente como puntos fijos de operadores} \\
    Sea L un reticulo completo cuyos puntos son conjuntos, (Podriamos decir $L\subseteq Pow(A)$, para algún conjunto $A$), ordenado bajo la relación de ordén de inclusión, y $F:L\rightarrow L$ una enfofunción en $L$, Los conjuntos:
    $$F_{ind} = \bigcap \{x|F(x)\subseteq x\}$$
    $$F_{coind} = \bigcup \{x|x\subseteq F(x)\}$$
    (El meet de de los puntos pre-fijos, y el join de los puntos post-fijos) son, respectivamente el conjunto definido inductiva por $F$ y definido coinductivamente por $F$.
\end{defi}

En el libro An Introduction to Inductive Definitions de Aczel, está la demostración de que tomanmdo el conjunto de reglas, para $\A\subseteq L$, $(A,y)\in\phi$ para $y\in F(A)$, tenemos que $I(\phi) = F_{ind}$, por lo que este es otro punto de vista para los conjuntos definidos inductivamente.  

\begin{propo}\textbf{Principio de inducción y coinduccion}\\
    para una endofinción monotona $F:L\rightarrow L$ sobre un reticulos tenemos
    \begin{center}
        $F(x)\leq x \rightarrow F_{ind}\leq x$ (Principio de inducción)\\
        $F(x)\geq x \rightarrow F_{coind}\geq x$ (Principio de coinducción)
    \end{center}
\end{propo}

Podemos volver a los ejemplos que estamos tratando. 

\begin{ejemplo} \textbf{El conjunto de los números Naturales, Como conjunto definidos inductivamente}\\
Sea $\varphi:Pow(X)\longrightarrow Pow(X)$,con $X$ un conjunto donce $0\in X$ y $X$ es cerrado para la operación sucesor, definimos:
$$\varphi(X)=\{0\} \cup \{s(x)\,\,|\,\,x\in X\}$$
Luego los naturales se ven definidos como $\N=\bigcap \{X \mid X\subseteq\varphi(X)\}$, el conjunto inductivamente definido. Por otra parte el conjunto coinductivamente definido es $X$.
\end{ejemplo}
    
\begin{ejemplo}\textbf{El conjunto de listas finitas de un conjunto A, Como conjunto definidos inductivamente}\\
Tomemos:\\ 
    $X$ como el conjunto de todos los strings finitos o infinitos con elementos del alfabeto $A\cup\{nill, \cdot, \rangle, \langle\}$, el funcional correspondiente $\varphi_{L_A}$ es:
    $$\varphi_{L_A}(T) = \{ nill\} \cup \{\langle a\rangle \cdot s \mid a\in A \land s \in T \}$$
Tenemos que el conjunto de listas finitas con elementos en $A$ es el conjunto inductivamente definido por el operador $\varphi_{L_A}$ y el conjunto de listas infinitas es el coinductivamente definido por este operador.
\end{ejemplo}

En el siguiente ejemplo podemos ver un patrón de cómo entender la equivalencia entre conjuntos inductivamente definidos a partir de reglas y a partir de puntos fijos en un reticulo completo.

\begin{ejemplo}\textbf{El conjunto de formulas de la lógica proposicional clásica $Form(CPL)$, Como conjunto definidos inductivamente}\\
    Dado un conjunto $V = \{p_1, p_2,...\}$ de variables proposicionales, $X$ como el conjunto de todos los strings finitos o infinitos con elementos del alfabeto $V\cup\{\lor, \lnot, \land, \rightarrow, \leftrightarrow, (, )\}$, el funcional correspondiente $\varphi$ es:
    $$\varphi(T) = V \cup \{(A) \# (B) \mid A , B\in T \,\,{y}\,\, \#\in \{\lor, \land, \rightarrow, \leftrightarrow\}\}\cup T \cup \{\lnot A|A\in T\}$$
    Que es un operador monotono sobre un latice completo.\\
    Tenemos finalmente que $Form(CPL) = \varphi_{ind}$, es el conjunto inductivamente definido por este operador, y los elementos aquó son justamente formulas.\\    
    Además, El conjunto coinductivamente definido ${C} = \bigcup \{T \mid T\subseteq\varphi(T)\}$, coincide con el conjunto de las formulas fintitas, y las generadas infinitamente a partir de estas reglas.
\end{ejemplo}



\newpage
\section{Referencias}
\begin{thebibliography}{9}
    \bibitem{Aczel}
    Aczel, P. (1997). 
    An Introduction to Inductive Definitions
    \bibitem{Davide Sangiorgi}
    Davide, S. (2011)
    Introduction to bisimulation and coinduccion
\end{thebibliography}
\end{document} 