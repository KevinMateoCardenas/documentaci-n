\documentclass[dvipsnames, 8pt]{beamer} % Cambia 10pt a 8pt para una fuente más pequeña
\usepackage[utf8]{inputenc}
\usepackage[spanish]{babel}             % Chapter -> Capítulo etc...
\usepackage[T1]{fontenc}                % Acentos como ö
\usepackage{lmodern}                    % Latin modern fonts
\usepackage{amsmath,amssymb,amsfonts,latexsym,stmaryrd}
\usepackage{graphicx, fancyhdr, wrapfig, tikz, textpos}
\usepackage{calligra}
\usepackage{ stmaryrd }
\usepackage{cite}
\usepackage{graphicx}
\usepackage[utf8]{inputenc}
\usepackage[T1]{fontenc}
\usepackage[spanish]{babel} % Establece el idioma español
\usepackage{csquotes} % Carga el paquete csquotes
\usepackage{graphicx} % Required for inserting images
\usepackage{listings}
\usepackage{xcolor}
\usepackage{hyperref}
\usepackage{geometry}
\usepackage{tikz}
\usetikzlibrary{shapes,arrows}
\usetikzlibrary{positioning}
\setlength{\parindent}{0.5in}
\usepackage{setspace}
\usepackage{amssymb}
\usepackage{amsthm}
\usepackage{ dsfont }
\usepackage{amsmath,amsfonts,amssymb}
\usepackage{tikz}
\usetikzlibrary{matrix}
\usetikzlibrary{graphs,graphs.standard}
\usepackage{etoolbox}

% Configuración del paquete hyperref
\hypersetup{
    colorlinks=true,
    linkcolor=black,
    filecolor=magenta,      
    urlcolor=gray,
}

\renewcommand{\baselinestretch}{1.5}

\theoremstyle{plain}

\newcommand{\Al}{(\mathcal{A},\mathds{F},\odot)}
\newcommand{\A}{\mathcal{A}}
\newcommand{\B}{\mathcal{B}}
\newcommand{\D}{\mathcal{D}}
\newcommand{\C}{\mathcal{C}}
\newcommand{\I}{\mathcal{I}}
\newcommand{\J}{\mathcal{J}}
\newcommand{\R}{\mathds{R}}
\newcommand{\N}{\mathbb{N}}
\newcommand{\Z}{\mathbb{Z}}
\newcommand{\fu}{f:D\longrightarrow \mathds{R}}
\newcommand{\fun}{f:[a,b]\longrightarrow \mathds{R}}
\newcommand{\E}{\mathcal{E}}
\newcommand{\F}{\mathds{F}}
\newcommand{\op}{``}
\newcommand{\cl}{''}
\newcommand{\po}{^}
\newcommand{\Q}{\matbbb{Q}}
\newcommand{\V}{\mathds{V}}
\newcommand{\T}{\mathds{T}}

\theoremstyle{definition}
\newtheorem{teo}{Teorema}
\newtheorem{defi}{Definición}
\newtheorem{obs}{Observación}
\newtheorem{ejemplo}{Ejemplo}

% Definición de colores estilo Universidad de Antioquia
\definecolor{udeaGreen}{RGB}{0, 104, 55} % Verde UdeA
\definecolor{udeaGray}{RGB}{88, 88, 90}  
% Verde Obscuro UdeA

% Aplicando colores a la presentación
\usetheme{Madrid}
\usecolortheme[RGB={217,217,217}]{structure}
\setbeamercolor{palette primary}{bg=udeaGreen,fg=white}
\setbeamercolor{title}{fg=white,bg=udeaGreen}
\setbeamercolor{frametitle}{fg=white,bg=udeaGreen}
\setbeamercolor{section number projected}{bg=udeaGreen,fg=white}
\setbeamercolor{item projected}{bg=udeaGreen,fg=white}
\setbeamersize{text margin left=10mm, text margin right=10mm}

\usefonttheme{professionalfonts}
\setbeamersize{text margin left=10mm, text margin right=10mm}

%------------------ INFORMACIÓN ------------------
\title[Definiciones Inductivas]{Definiciones Inductivas}
\author[Kevin Cárdenas]{\large Kevin Mateo Cárdenas Gallego\vspace{1cm}}
\institute[Universidad de Antioquia]
        {\large 
        Asesor: Juan Carlos Agudelo\\
        Instituto de Matemáticas\\
        Universidad de Antioquia}
\date[Noviembre 2023]{}
\logo{\includegraphics[scale=0.1]{logosimbolo-horizontal-png.png}}
%------------------ INICIO DEL DOCUMENTO ------------------
\begin{document}

\setbeamertemplate{background}
{
	\parbox[c][\paperheight]{\paperwidth}
	{
		\vfill
        \begin{center}
            \begin{tikzpicture}
                \node[opacity=.1]
                {
                    \includegraphics[scale=0.4]{UDEA.png}
                };
            \end{tikzpicture}
        \end{center}
	}
}

\AtBeginSection[]
{
\begin{frame}<beamer>{Tabla de contenido}
\tableofcontents[currentsection,currentsubsection]
\end{frame}
}
\AtBeginSubsection[]
{
\begin{frame}<beamer>{Tabla de contenido}
\tableofcontents[currentsection,currentsubsection]
\end{frame}
}

%------------------ TITLE PAGE ------------------
\begin{frame}
	\titlepage
\end{frame}

\addtobeamertemplate{frametitle}{}{
\begin{textblock*}{100mm}(0.86\textwidth,-1.1cm)
	\includegraphics[scale=0.3]{UDEA.png}
\end{textblock*}}

\begin{frame}{Tabla de contenido}
    \tableofcontents
\end{frame}

\begin{frame}{Contexto Histórico}
    \begin{itemize}
        \item La teoría de tipos surge como una respuesta a las paradojas encontradas en las matemáticas y la lógica a finales del siglo XIX y principios del XX, en particular, la paradoja de Russell. Esta paradoja demostraba que la teoría de conjuntos, como se conocía, podía conducir a contradicciones.\pause
        \item Bertrand Russell introdujo la teoría de tipos en 1908 para superar estas paradojas. En su obra, \emph{Principia Mathematica}, coescrita con Alfred North Whitehead, Russell propuso que los elementos de cualquier conjunto deberían ser de un tipo diferente al conjunto mismo, evitando así la formación de conjuntos autorreferenciales.\pause
        \item La teoría de tipos experimentó una evolución significativa a lo largo del siglo XX. En la década de 1970, Per Martin-Löf introdujo la teoría de tipos intuicionista, que incorporaba principios de la lógica constructivista y tenía implicaciones profundas tanto para las matemáticas como para la informática, especialmente en el diseño de lenguajes de programación y sistemas de prueba formal.
    \end{itemize}
\end{frame}

\begin{frame}{Introducción}
    \begin{ejemplo} [El conjunto de los números naturales $\N$]
        El conjunto $\N$ de los números naturales es el \emph{menor conjunto} tal que:\pause
        \begin{enumerate}
            \item $0 \in \N$.
            \item Si $n \in \N$, entonces $s(n )\in \N$.
        \end{enumerate}
    \end{ejemplo}\pause
    \begin{ejemplo}[El conjunto de listas finitas sobre un conjunto dado]
        Dado un conjunto $A$, el conjunto $Finlist(A)$ es el menor conjunto tal que:\pause
        \begin{enumerate}
            \item $nill \in Finlist(A)$.\pause
            \item Si $S \in Finlist(A)$ y $a \in A$ , entonces $\langle a \rangle\cdot S\in Finlist(A)$.
        \end{enumerate}
    \end{ejemplo}
\end{frame}

\section{Formalización a partir de conjuntos de reglas}
\begin{frame}{Formalización a partir de conjuntos de reglas}
    \begin{defi}[Reglas, conjunto $\phi-$cerrado]
        \begin{enumerate}
            \item Una \emph{regla} es un par $(X,x)$, donde $X$ es un conjunto, llamado \emph{conjunto de premisas}, y $x$ es la \emph{conclusión}.\pause
            \item Si $\phi$ es un conjunto de reglas, entonces un conjunto $A$ es \emph{$\phi - cerrado$} si para toda regla $(X,x) \in \phi$ si se tiene que $X \subseteq A$ implica $x \in A$.\pause
            \item Dado $\phi$ un conjunto de reglas, se define \emph{$I(\phi)$ como el conjunto inductivamente definido por $\phi$}, dado por:
            $$I(\phi) = \bigcap \{A : A \hspace{2mm}\phi{-cerrado}\}.$$
        \end{enumerate}
    \end{defi}
\end{frame}

\begin{frame}{Formalización a partir de conjuntos de reglas}
    \begin{ejemplo}[$\N$ a partir de un conjunto de reglas]
        El conjunto de los números naturales $\N$ se puede definir como un conjunto inductivamente definido a partir de un conjunto de reglas $\phi$ dado por: \pause
        \begin{enumerate}
            \item $(\emptyset, 0) \in \phi$.\pause
            \item $(\{n\}, s(n)) \in \phi$.
        \end{enumerate}
    \end{ejemplo}
\end{frame}

\begin{frame}{Formalización a partir de conjuntos de reglas}
    \begin{ejemplo}[$F(CPL)$ a partir de un conjunto de reglas]
        Dado $V$ un conjunto de variables proposicionales, sea $\phi$ el conjunto con las siguientes reglas:
        \begin{enumerate}
            \item $(\emptyset, p_i)$ para cada $p_i \in V$.\pause
            \item $(\{A\}, \neg A)$.
            \item $(\{A, B\}, (A \lor B))$.
            \item $(\{A, B\}, (A \land B))$.
            \item $(\{A, B\}, (A \rightarrow B))$.
            \item $(\{A, B\}, (A \leftrightarrow B))$.
        \end{enumerate}
    \end{ejemplo}
\end{frame}

\begin{frame}{Formalización a partir de conjuntos de reglas}
    \begin{ejemplo}[$Finlist(A)$ a partir de un conjunto de reglas]
        Dado un conjunto $A$, y $\phi$ el conjunto que contiene las siguientes reglas:
        \begin{enumerate}
            \item $(\emptyset, nill)$.\pause
            \item $(\{S\},\langle a \rangle\cdot S)$, donde $a \in A$.
        \end{enumerate}\pause
    \end{ejemplo}
    \begin{teo}[Demostraciones por inducción]
        Sea $A$ un conjunto inductivamente definido a partir de un conjunto de reglas $\phi$, y una propiedad sobre este 
        $$\varphi : A \rightarrow \{true, false\}.$$ 
        Demostrar que $\varphi^{-1}(\{true\}) = A$, se reduce a demostrar que 
        \begin{center}
            $\varphi^{-1}(\{true\})$ es $\phi$-cerrado.
        \end{center}
    \end{teo}
\end{frame}

\section{Formalización a partir de operadores monótonos}
\begin{frame}{Formalización a partir de operadores monótonos}
    \begin{defi}[CPO]
        Un \emph{conjunto parcialmente ordenado o CPO} $(A,\leq)$, es un conjunto equipado $A$ con una relación de orden parcial $\leq$, es decir, \emph{reflexiva, antisemítica y transitiva}.
    \end{defi}\pause

    \begin{defi}
        Dado un CPO $(L, \leq)$ y un operador $F: L \rightarrow L$, definimos:\pause
        \begin{itemize}
            \item $F$ es \emph{monótono} si preserva el orden parcial.\pause
            \item Los \emph{puntos pre-fijos} de $F$ son los elementos $x \in L$ tales que $F(x) \leq x$. De manera similar, los \emph{puntos post-fijos} de $F$ son los elementos $x \in L$ tales que $x \leq F(x)$.\pause
            \item Los \emph{puntos fijos} de $F$ son los elementos $x \in L$ tales que $F(x) = x$.
        \end{itemize}
    \end{defi}\pause
    \begin{defi}[Retículo completo]
        Un retículo completo es un conjunto parcialmente ordenado $(L,\leq)$ en el que todos los subconjuntos tienen supremo.\pause
    \end{defi}
\end{frame}

\begin{frame}{Formalización a partir de operadores monótonos}
    \begin{defi}[Conjuntos definidos inductivamente]
        \begin{itemize}
            \item Sea L un retículo completo cuyos puntos son conjuntos, ordenado bajo la relación de inclusión, y $F:L\rightarrow L$ un operador monótono sobre $L$.\\ \pause
            \emph{El conjunto definido inductivamente por $F$} es el conjunto:
            $$F_{ind} = \bigcap \{x \hspace{2mm}:F(x)\subseteq x\}.$$
        \end{itemize}
    \end{defi}\pause
    \begin{teo}[De operadores monótonos a conjuntos de reglas]
        Sean $A$ un conjunto, $(L, \subseteq)$ un retículo completo, con $L\subseteq P(A)$, $F: L \to L$ un operador monótono sobre $L$, entonces existe un conjunto de reglas $\phi$ tal que $F_{ind} = I(\phi)$.
    \end{teo}
\end{frame}

\begin{frame}{Formalización a partir de operadores monótonos}
    \begin{ejemplo}[$\N$ a partir de un operador monótono]
        Sea $X$ el conjunto de cadenas finitas o infinitas de elementos en el alfabeto $\{0, s, (, )\} $, sea $\varphi:P(X)\longrightarrow P(X)$ definido por:
        $$\varphi(T)=\{0\} \cup \{s(x)\hspace{2mm}:\hspace{2mm}x\in T\}.$$
        Se tiene que los naturales se pueden definir como
        $\N=\bigcap \{x \hspace{2mm}: \varphi(x)\subseteq x\}$.
    \end{ejemplo}\pause
    \begin{ejemplo}[$FinList_{A}$ a partir de un operador monótono]
        Sea $X$ el conjunto de todas las cadenas finitas o infinitas con elementos del alfabeto $A\cup\{nill, \cdot, \rangle, \langle\}$, $(P(X),\subseteq)$ el retículo, y el operador correspondiente $\varphi_{L_A}$ es:
        $$\varphi_{L_A}(T) = \{ nill\} \cup \{\langle a\rangle \cdot s \hspace{2mm}: a\in A \land s \in T \}.$$
\end{ejemplo}\end{frame}

\begin{frame}{Formalización a partir de operadores monótonos}
    \begin{ejemplo}[$F(CPL)$ a partir de un operador monótono]
        Dado un conjunto de variables proposicionales
        $$V = \{p_1, p_2,...\}.$$ \pause
        $X$ el conjunto de todas las cadenas finitas o infinitas con elementos del alfabeto $V\cup\{\lor, \lnot, \land, \rightarrow, \leftrightarrow, (, )\}$, el retículo:
        $$(P(X),\subseteq).$$\pause
        Y el operador correspondiente $\varphi$ es:
        \begin{align*}
            \varphi(T) = V &\cup \{(A \# B) \hspace{2mm}: (A , B\in T) \land (\#\in \{\lor, \land, \rightarrow, \leftrightarrow\})\}\\
                            &\cup \{\lnot A \hspace{2mm}:A\in T\}
        \end{align*}
    \end{ejemplo}
\end{frame}

\section{Esquema de tipos inductivos en teoría de tipos}
\begin{frame}{Esquema de tipos inductivos en teoría de tipos}
    \begin{defi}[Expresiones]
        Se usa notación ordinaria, pero se omite mencionar restricciones de variables.
        \begin{itemize}
            \item Expresiones de conjuntos $A ::= \Pi x:A_{0}.A_{1}[x].$ \pause
            \item Expresiones de elementos $a ::= x \hspace{2mm}|\hspace{2mm} \lambda x:A.a[x] \hspace{2mm}|\hspace{2mm} a_1(a_0).$ \pause
            \item Expresiones de contextos $\Gamma ::= \epsilon \hspace{2mm}|\hspace{2mm} \Gamma, x:A.$ \pause
            \item Expresiones de juicios $J ::= \Gamma \hspace{2mm}|\hspace{2mm} \Gamma \vdash A \hspace{2mm}|\hspace{2mm} \Gamma \vdash a:A \hspace{2mm}|\hspace{2mm} \Gamma \vdash A = A' \hspace{2mm}|\hspace{2mm} \Gamma \vdash a = a':A.$
        \end{itemize}
    \end{defi}
\end{frame}

\begin{frame}{Esquema de tipos inductivos en teoría de tipos}
    \begin{itemize}
        \item Las reglas de inferencia para el sistema de tipos de Martín-Löf son las siguientes:\pause
    \end{itemize}
    $$\epsilon \hspace{2mm} contexto \hspace{6mm} \frac{\Gamma\hspace{2mm} contexto \hspace{4mm} \Gamma \vdash A \hspace{2mm} set}{\Gamma,x:A \hspace{2mm} contexto}.$$
    \vspace{0.3mm}
    $$\frac{\Gamma \vdash A \hspace{2mm} set}{\Gamma \vdash A = A} \hspace{6mm} \frac{\Gamma \vdash a : A}{\Gamma \vdash a = a : A}$$
    \vspace{0.3mm}
    $$\frac{\Gamma \vdash A = A'}{\Gamma \vdash A' = A} \hspace{6mm} \frac{\Gamma \vdash a = a' : A}{\Gamma \vdash a' = a : A}$$
    \vspace{0.3mm}
    $$\frac{\Gamma \vdash A = A' \hspace{8mm} \Gamma \vdash A' = A''}{\Gamma \vdash A = A''} \hspace{6mm} \frac{\Gamma \vdash a = a' : A \hspace{8mm} \Gamma \vdash a' = a'' : A}{\Gamma \vdash a = a'' : A}$$
    \vspace{0.3mm}
    $$\frac{\Gamma \vdash A = A' \hspace{8mm} \Gamma \vdash a : A}{\Gamma \vdash a : A'} \hspace{6mm} \frac{\Gamma \vdash A = A' \hspace{8mm} \Gamma \vdash a = a' : A}{\Gamma \vdash a = a'' : A'}$$
    \vspace{0.3mm}
    $$\frac{\Gamma\vdash A \hspace{2mm} set}{\Gamma, x:A \vdash x:A}$$
    \vspace{0.3mm}
    $$\frac{\Gamma\vdash A_{0} \hspace{2mm} set \hspace{8mm} \Gamma\vdash A_{1} \hspace{2mm} set}{\Gamma, x:A_{0} \vdash A_{1} \hspace{2mm}set} \hspace{8mm} \frac{\Gamma\vdash A_{0} \hspace{2mm} set \hspace{8mm} \Gamma\vdash a:A_{1}}{\Gamma, x:A_{0} \vdash a:A_{1}}$$
\end{frame}

\begin{frame}{Esquema de tipos inductivos en teoría de tipos}
    \begin{itemize}
        \item Reglas de formación de tipos para el producto cartesiano de una familia de conjuntos:\pause
    \end{itemize}
        \[
            \frac{\Gamma\vdash A_{0} \hspace{2mm} set \hspace{4mm} \Gamma, x:A_{0}\vdash A_{1}[x] \hspace{2mm} set}{\Gamma \vdash \Pi x:A_{0}.A_{1}[x] \hspace{2mm}set}
        .\]
        \vspace{2mm}
        \[
            \frac{\Gamma, x:A_{0}\vdash a[x]:A_{1}}{\Gamma \vdash \lambda x:A_{0}.a[x] : \Pi x:A_{0}.A_{1}[x]} 
        .\]
        \vspace{2mm}
        \[
            \frac{\Gamma \vdash a_{1}: \Pi x:A_{0}.A_{1}[x] \hspace{3mm} \Gamma\vdash a_{0}:A_{0}}{\Gamma \vdash a_{1}(a_{0}) : A_{1}[a_{0}]}
        .\]
\end{frame}

\begin{frame}{Esquema de tipos inductivos en teoría de tipos}
    \begin{itemize}
        \item Reglas de igualdad de tipos para el producto cartesiano de una familia de conjuntos:\pause
    \end{itemize}
        \[
            \frac{\Gamma\vdash A_{0} = A_{0}' \hspace{4mm} \Gamma, x:A_{0}\vdash A_{1}[x] = A_{1}'[x]}{\Gamma \vdash \Pi x:A_{0}.A_{1}[x] = \Pi x:A_{0}'.A_{1}'[x]}
        .\]
        \vspace{2mm}
        \[
            \frac{\Gamma, x:A_{0}\vdash a[x] = a'[x] : A_{1}}{\Gamma \vdash \lambda x:A_{0}.a[x] = \lambda x:A_{0}.a'[x] : \Pi x:A_{0}.A_{1}[x]}
        .\]
        \vspace{2mm}
        \[
            \frac{\Gamma \vdash a_{1} = a_{1}' : \Pi x:A_{0}.A_{1}[x] \hspace{3mm} \Gamma \vdash a_{0} = a_{0}':A_{0}}{\Gamma \vdash a_{1}(a_{0}) = a_{1}'(a_{0}'): A_1[a_{0}]}
        .\]
        \vspace{2mm}
        \[
            \frac{\Gamma.x:A_{0}\vdash a_{1}[x]: A_{1}[x] \hspace{4mm} \Gamma\vdash a_{0} : A_{0}}{\Gamma\vdash(\lambda x:A_{0}.a_1[x])(a_{0}) = a_{1}(a_{0}) : A_{1}[a_{0}]}
        .\]
\end{frame}

\begin{frame}{Esquema de tipos inductivos en teoría de tipos}
    \begin{defi}[Telescopios]
        Un \emph{telescopio} es una secuencia de tipos dependientes \(T_{1}, \ldots, T_{n}\), donde cada \(T_{i}\) es un tipo dependiente de los parámetros \(x_{1}:T_{1}, \ldots, x_{i-1}:T_{i-1}\) para todo \(i \in \{2, \ldots, n\}\).
    \end{defi}\pause
    \begin{itemize}
        \item Se tiene la sintaxis y las reglas del $\lambda$-cálculo con tipos dependientes. Llamamos a esta teoría $T_{0}$. \pause
        \item $T_{0}$ puede ser extendida sucesivamente obteniendo las teorías $T_{1}, T_{2}, \ldots$ de la siguiente manera:\pause
        \item $T_{0}$ es la teoría de tipos dependientes que se acaba de definir. $T_{i+1}$ es la teoría $T_{i}$ extendida con un nuevo tipo de datos $I_{i}$.\pause
        \item Se tiene que $J$ abrevia $\Gamma\vdash J$, en una teoría $T$.
    \end{itemize}
\end{frame}

\begin{frame}{Esquema de tipos inductivos en teoría de tipos}
    \begin{itemize}
    \item Reglas de formación:
    \[P \hspace{2mm} set, \hspace{4mm} P=P\]\pause
    \item Reglas de introducción i-ésima:
      $$\frac{as::Gs_{i} \hspace{8mm} (b_{k}:Hs_{ik}[as]\to P)_{k}}{intro_{i}(as,(b_{k})_{k}) :P}.$$
      $$\frac{as=as'::Gs_{i} \hspace{8mm} (b_{k}=b'_{k}:Hs_{ik}[as]\to P)_{k}}{intro_{i}(as,(b_{k})_{k}) = intro_{i}(as,(b´'_{k})_{k}) :P}.$$\pause
    Donde\pause
        \begin{enumerate}
            \item $Gs_{i}$ es un telescopio relativo a $T$;\pause
            \item $Hs_{ik}[xs]$ es un telescopio relativo a $T$ en el contexto $xs::Gs_{i}$ para cada $k$.
        \end{enumerate}
    \end{itemize}
\end{frame}

\begin{frame}{Esquema de tipos inductivos en teoría de tipos}
    \begin{itemize}
        \item En ese sentido, se tiene la siguiente espresión de conjunto:\pause
        $$A::=P.$$\pause
        \item y las siguientes expresiones de elementos:\pause
        $$a::= intro_{i}(as,(b_k)_k).$$
    \end{itemize}
    
\end{frame}

\begin{frame}{Esquema de tipos inductivos en teoría de tipos}
    \begin{ejemplo}[Naturales]
        Definimos:\pause
        \begin{itemize}
            \item Regla de formación:\pgfshapeborderusesincirclefalse
            \begin{center}
                $\N$ set\\
                $\N=\N$.
            \end{center}\pause
            \item Reglas de introducción:\pause
            $$\frac{}{0:\N} \hspace{16mm} \frac{}{0=0:\N}.$$
            $$\frac{n:\N}{s(n):\N} \hspace{16mm} \frac{n = n':\N}{s(n) = s(n'):\N}.$$
        \end{itemize}
    \end{ejemplo}
\end{frame}

\begin{frame}{Esquema de tipos inductivos en teoría de tipos}
    \begin{ejemplo}[Listas finitas sobre un conjunto dado]
        Dado $A :set$ definimos\pause
        \begin{itemize}
            \item Regla de formación:\pause
            \begin{center}
                $FinList_A$ set\\
                $FinList_A=FinList_A$.\pause
            \end{center}
            \item Reglas de introducción:\pause
            $$\frac{}{nill:FinList_A} \hspace{16mm} \frac{}{nill=nill:FinList_A}.$$
            $$\frac{a:A\hspace{8mm}l:FinList_A}{\langle a\rangle\cdot l:FinList_A} \hspace{16mm} \frac{a=a':A\hspace{8mm}l=l':FinList_A}{\langle a\rangle\cdot l=\langle a'\rangle\cdot l':FinList_A}.$$
        \end{itemize}
    \end{ejemplo}    
\end{frame}

\begin{frame}{Esquema de tipos inductivos en teoría de tipos}
    \begin{ejemplo}[El conjunto de formulas de la lógica proposicional clásica]
        Dado $V :set$ el conjunto de variables proposicionales, definimos\pause
        \begin{itemize}
            \item 
            Regla de formación:\pause
            \begin{center}
                $F(CPL)$ set.\\
                $F(CPL)=F(CPL)$.\pause
            \end{center}
            \item Reglas de introducción:
        \end{itemize}
    \end{ejemplo}
\end{frame}

\begin{frame}{Esquema de tipos inductivos en teoría de tipos}
    \begin{ejemplo}[El conjunto de formulas de la lógica proposicional clásica]
        $$\frac{a:V}{a:F(CPL)} \hspace{6mm} \frac{a=a':V}{a=a':F(CPL)}.$$
        $$\frac{g:F(CPL)}{(\lnot g):F(CPL)} \hspace{6mm} \frac{g=g':F(CPL)}{(\lnot g)=(\lnot g'):F(CPL)}.$$
        $$\frac{g:F(CPL)\hspace{4mm}f:F(CPL)}{(g\lor f):F(CPL)} \hspace{6mm} \frac{g=g':F(CPL)\hspace{4mm}f=f':F(CPL)}{(g\lor f)=(g'\lor f'):F(CPL)}.$$
        $$\frac{g:F(CPL)\hspace{4mm}f:F(CPL)}{(g\land f):F(CPL)} \hspace{6mm} \frac{g=g':F(CPL)\hspace{4mm}f=f':F(CPL)}{(g\land f)=(g'\land f'):F(CPL)}.$$
        $$\frac{g:F(CPL)\hspace{4mm}f:F(CPL)}{(g\rightarrow f):F(CPL)} \hspace{6mm} \frac{g=g':F(CPL)\hspace{4mm}f=f':F(CPL)}{(g\rightarrow f)=(g'\rightarrow f'):F(CPL)}.$$
        $$\frac{g:F(CPL)\hspace{4mm}f:F(CPL)}{(g\leftrightarrow f):F(CPL)} \hspace{6mm} \frac{g=g':F(CPL)\hspace{4mm}f=f':F(CPL)}{(g\leftrightarrow f)=(g'\leftrightarrow f'):F(CPL)}.$$
    \end{ejemplo}
\end{frame}

\section{Interpretación de MLTT en Teoría de Conjuntos}

\begin{frame}{Interpretación de MLTT en Teoría de Conjuntos}
    \begin{itemize}
        \item $\llbracket a \rrbracket\rho$ denota la interpretación de la expresión $a$ bajo la asignación $\rho$. Esta asignación es una función que asigna un conjunto a cada variable $x$ en una lista finita de variables que contenga todas las variables libres de $a$.\pause
        \item Para un tipo inductivo $P$, tome el siguiente conjunto de reglas:\pause    
        \begin{align*}
            \Phi = \bigcup \left\{\left(X, x\right)\hspace{2mm}: us\in\llbracket Gs_{i}\rrbracket, (v_k\in \llbracket Hs_{ik}[xs]\rrbracket_{xs}^{us}\to U)_{k} \right\}.
        \end{align*}
        Donde.
        \begin{enumerate}
            \item $ X = \bigcup_{k} ran(v_{k})$, $x = \langle |intro_{i}|, us, (v_{k})_{k})\rangle$.\pause
            \item $\llbracket\cdot\rrbracket$ es la interpretación de tipos y $\llbracket\cdot\rrbracket_{xs}^{us}$ es la interpretación de telescopios.\pause
            \item $ \langle |intro_{i}|, us, (v_{k})_{k})\rangle$ representa una codificación de $intro_{i}(us,(v_{k})_{k})$. Donde $|intro_{i}|$ es el código de $intro_{i}$.
        \end{enumerate}
    \end{itemize}
\end{frame}

\begin{frame}{El conjunto $\N$}
    \begin{itemize}
        \item Formalmente hablando $U = V_{\alpha}$, donde $V_\alpha$ es el conjunto generado en la llamada jerarquía acumulativa de conjuntos antes del estado $\alpha$, para un ordinal $\alpha$ suficientemente grande.\pause
        \item Por simplicidad se tomará a $U$ como el conjunto de cadenas finitas o infinitas sobre un alfabeto que contenga la sintaxis necesaria para representar los términos en el conjunto de reglas.\pause
    \end{itemize}
    \begin{ejemplo}
        \begin{itemize}
            \item (Reglas de introducción)\pause
                  $$0:\N.$$
                  $$s:\N\to\N.$$
        \end{itemize}\pause
        Entonces el conjunto de reglas bajo la interpretación es:\pause
        \begin{align*}
            \Phi = & \{(\emptyset, 0)\}\\
                   & \cup \{(\{n\}, \langle s\rangle\cdot n)\hspace{2mm}: n \in U\}
        .\end{align*}
    \end{ejemplo}
\end{frame}

\begin{frame}{Conclusiones}
    \begin{itemize}
        \item Se ha mostrado como formalizar definiciones inductivas a partir de conjuntos de reglas y operadores monótonos.\pause
        \item Se ha mostrado como formalizar tipos inductivos en teoría de tipos de Martín-Löf.\pause
        \item Por último, se ha mostrado como interpretar tipos inductivos en teoría de conjuntos.
    \end{itemize}
\end{frame}

\begin{frame}{Referencias}
    \begin{thebibliography}{9}
        \bibitem{Davide Sangiorgi}
        \textcolor{black}{Sangiorgi, D. (2011)Introduction to Bisimulation and Coinduction.
        Cambridge University Press.
        }
        \bibitem{Peter Dybjer}
        \textcolor{black}{Dybjer, P (1991)
        Inductive Sets and Families in Martín-Lof's Types Theory and theis Set-Theoretic Semantics. In G. Huet and G. Plotkin, editors, Logical Frameworks, pages 280-306. Cambridge University Press.
        }
        \bibitem{Dybjer and Chalmer}
        \textcolor{black}{Dybjer, P. Chalmer, T. H. (2005)
        Lectures on Inductive and Recursive Definitions in Constructive Type Theory. TYPES Summer School, Göteborg, August 2005.
        }
        \bibitem{Aczel}
        \textcolor{black}{Aczel, P. (1997)
        An Introduction to Inductive Definitions. In Studies in Logic and the Foundations of Mathematics, volume 90. pages 739-782.
        }
    \end{thebibliography}
\end{frame}

\begin{frame}
    \begin{center}
        {\Huge\calligra ¡Gracias!}
    \end{center}
\end{frame}

\end{document}
