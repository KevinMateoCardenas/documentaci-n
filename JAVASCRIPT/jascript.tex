\documentclass[executivepaper]{article}

\usepackage{graphicx}
\usepackage[utf8]{inputenc}
\usepackage[T1]{fontenc}
\usepackage[spanish]{babel} % Establece el idioma español
\usepackage{csquotes} % Carga el paquete csquotes
\usepackage{graphicx} % Required for inserting images
\usepackage{listings}
\usepackage{xcolor}
\usepackage{hyperref}
\usepackage[left=1.00cm, right=1.00cm, top=2.00cm, bottom=2.00cm]{geometry}
\usepackage{tikz}
\usetikzlibrary{shapes,arrows}
\usetikzlibrary{positioning}
\setlength{\parindent}{0.5in}
\usepackage{setspace}
\doublespacing

\lstset{
    inputencoding=utf8,
    language=Java,
    basicstyle=\ttfamily,
    columns=fullflexible
}

% Define colores para el código
\definecolor{codegreen}{rgb}{0,0.6,0}
\definecolor{codegray}{rgb}{0.5,0.5,0.5}
\definecolor{codepurple}{rgb}{0.58,0,0.82}
\definecolor{backcolour}{rgb}{0.95,0.95,0.92}

% Configuración de lstlisting
\lstdefinestyle{mystyle}{
    backgroundcolor=\color{backcolour},   
    commentstyle=\color{codegreen},
    keywordstyle=\color{magenta},
    numberstyle=\tiny\color{codegray},
    stringstyle=\color{codepurple},
    basicstyle=\ttfamily\footnotesize,
    breakatwhitespace=false,         
    breaklines=true,                 
    captionpos=b,                    
    keepspaces=true,                 
    numbers=left,                    
    numbersep=5pt,                  
    showspaces=false,                
    showstringspaces=false,
    showtabs=false,                  
    tabsize=2
}

% Configuración del paquete hyperref
\hypersetup{
    colorlinks=true,
    linkcolor=black,
    filecolor=magenta,      
    urlcolor=gray,
}

\lstset{style=mystyle}

\title{Informe sobre JAVASCRIPT}
\author{Kevin Cárdenas}

\begin{document}

\begin{titlepage}
    \begin{center}
        {\Huge \textbf{JAVASCRIPT}}
        \\[18cm]

        \large\emph{Autor:}\\
        Kevin Cárdenas.
        \\[1cm]
        {\large 2023}
    \end{center}
\end{titlepage}

\newpage
\tableofcontents
\newpage

\section{Introducción}

\subsection*{¿Qué es JavaScript?}
JavaScript es un lenguaje de programación de alto nivel y dinámico que se utiliza principalmente en el desarrollo web para agregar interactividad a las páginas web. Es un lenguaje interpretado, lo que significa que se ejecuta en tiempo de ejecución sin necesidad de ser compilado previamente.

\subsection*{¿Por qué se llama Java Script?}
Aunque comparten algunas similitudes en la sintaxis, JavaScript no está relacionado con el lenguaje de programación Java. Inicialmente, se llamaba Mocha y luego se cambió a LiveScript antes de adoptar el nombre de JavaScript para capitalizar el éxito de Java en ese momento.

\subsection*{¿Cómo funcionan los motores?}
Los motores de JavaScript son programas que interpretan y ejecutan el código JavaScript en un navegador. Los motores más comunes son V8 (usado por Google Chrome), SpiderMonkey (usado por Mozilla Firefox) y JavaScriptCore (usado por Safari). Los motores utilizan técnicas de optimización avanzadas, como la compilación JIT (Just-In-Time), para hacer que la ejecución del código sea lo más rápida posible.

\subsection*{¿Qué puede hacer JavaScript en el navegador?}
JavaScript es un lenguaje de programación muy versátil que se puede utilizar para crear una variedad de funciones en el navegador, incluyendo:

\begin{itemize}
\item Manipulación del DOM: JavaScript puede manipular la estructura de una página web y los elementos que la componen.
\item Eventos: JavaScript puede responder a eventos en una página web, como clics de botón o desplazamiento.
\item Validación de formularios: JavaScript puede validar formularios para garantizar que los usuarios ingresen información correcta.
\item Animaciones y efectos visuales: JavaScript se puede utilizar para crear animaciones y efectos visuales, como desvanecimientos y transiciones.
\end{itemize}

\subsection*{¿Qué NO PUEDE hacer JavaScript en el navegador?}
Aunque JavaScript es un lenguaje de programación muy poderoso, hay algunas cosas que no puede hacer en el navegador, como:

\begin{itemize}
\item Acceder a archivos locales: JavaScript no puede leer ni escribir archivos locales en la computadora del usuario sin permiso explícito del usuario.
\item Acceder a la webcam y al micrófono: JavaScript no puede acceder a la cámara web ni al micrófono del usuario sin su permiso explícito.
\item Realizar operaciones en segundo plano: JavaScript no puede realizar operaciones en segundo plano sin el uso de técnicas avanzadas como Web Workers.
\end{itemize}

\subsection*{¿Qué hace que JavaScript sea único?}
JavaScript es único porque es uno de los pocos lenguajes de programación que se puede ejecutar en un navegador web sin la necesidad de ningún plugin externo. Además, JavaScript es muy versátil y se puede utilizar para crear una amplia variedad de aplicaciones y efectos visuales. También es uno de los lenguajes de programación más populares en el mundo y tiene una gran comunidad de desarrolladores y recursos disponibles en línea.

\newpage
\section{Manuales y especificaciones}

\end{document}