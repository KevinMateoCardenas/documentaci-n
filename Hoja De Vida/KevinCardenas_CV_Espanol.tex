\documentclass[11pt,a4paper,sans]{moderncv}        
\moderncvstyle{casual}   
\moderncvcolor{blue} 
\usepackage[T1]{fontenc}
\usepackage[utf8]{inputenc} 
\usepackage[scale=0.75]{geometry}
\setlength{\footskip}{122.40004pt}                
\name{Kevin Mateo}{Cárdenas Gallego}
\title{Curriculum vitae}   
\address{CALLE 72 NO 47-34 (int 102)}{Medellín}{Colombia}
\phone[mobile]{+57~(314)~881~02~04}  
\email{kevin.cardenas1@udea.edu.co}
\photo[90pt][0.4pt]{kevin.jpg} 
\social[orcid]{0000-0001-6278-7353}
\social[CvLAC]{\href{https://scienti.minciencias.gov.co/cvlac/visualizador/generarCurriculoCv.do?cod_rh=0001980317}{CvLAC}}
\begin{document}
\makecvtitle

\section{Educación}
\cventry{2018--2023}{Matemático}{Universidad de Antioquia}{Medellín}{\textit{Pregrado}}{Kevin Mateo Cárdenas Gallego culminó los créditos del programa de Matemáticas en la Universidad de Antioquia.}

\section{Experiencia Laboral}
\cventry{2022--Actualidad}{Desarrollador}{Quind SAS}{}{}{Se unió a Quind SAS en diciembre de 2022, trabajando en el desarrollo de soluciones en Python y C\#. Durante su formación, estudió SQL, Programación Orientada a Objetos (POO), programación funcional, patrones de arquitectura de software y diversas herramientas para el desarrollo de soluciones específicas.}
\cventry{2022-2023}{Monitor}{Universidad de Antioquia}{}{}{Fue monitor de las asignaturas de Cálculo Integral, Cálculo Vectorial, Geometría Vectorial y Álgebra Lineal.}

\section{Idiomas}
\cvitemwithcomment{Español}{Avanzado}{Lengua materna.}
\cvitemwithcomment{Inglés}{Intermedio}{Capacidad para leer y escribir con fluidez, así como habilidad para hablar y escuchar, aunque con cierta dificultad.}

\section{Intereses}
\cvitem{Computación}{Apasionado por la computación y la programación, considera realizar estudios de posgrado tras concluir el pregrado.}
\cvitem{Matemáticas}{En proceso de finalizar el pregrado en Matemáticas en la Universidad de Antioquia, con gran pasión por el conocimiento y el aprendizaje, considera continuar con estudios de posgrado.}
\cvitem{Música}{Aficionado a tocar la guitarra.}
\cvitem{Lectura}{Dedica al menos una hora al día a leer sobre temas de matemáticas, computación y otros tipos de información que encuentra interesantes.}

\section{Matriz de Habilidades}
\cvskillentry*{}{4}{Python}{4}{Kevin Mateo ha estudiado análisis numérico y los paradigmas de programación mencionados anteriormente en este lenguaje.}
\cvskillentry{}{4}{C\#}{4}{Apto para realizar programas en este lenguaje.}
\cvskillentry{}{4}{SQL}{4}{Capaz de realizar, estudiar, analizar, normalizar y consultar bases de datos relacionales y no relacionales; ha utilizado PostgreSQL anteriormente.}
\cvskillentry{}{4}{GIT y GitHub}{4}{Habilidad para utilizar GIT y GitHub para control de versiones y desarrollo colaborativo.}
\cvskillentry{}{3}{React JS}{3}{Desarrolla aplicaciones web con este framework; experiencia en sistemas de diseño como Material UI y Spectrum.}
\cvskillentry{}{3}{TypeScript}{3}{Desarrolla aplicaciones web con este lenguaje.}
\cvskillentry{}{3}{JavaScript}{3}{Desarrolla aplicaciones web con este lenguaje.}
\cvskillentry{}{4}{Docker}{4}{Capaz de utilizar Docker para el desarrollo de aplicaciones.}
\cvskillentry{}{5}{Computación en General}{5}{Durante su pregrado, estudió temas como computabilidad, métodos numéricos, teoría de tipos, asistentes de pruebas y teoría de funciones recursivas, desde máquinas de Turing hasta autómatas celulares. Esto le ha dotado de habilidades variadas útiles para la programación o diseño de algoritmos y, en general, para la resolución de problemas. Ha estudiado paradigmas de programación como la funcional y la orientada a objetos, y puede aplicar estos paradigmas en distintos lenguajes para ofrecer soluciones efectivas y prácticas.}
\cvskillentry{}{4}{Matemáticas}{4}{Durante su pregrado, estudió cursos como cálculo, álgebra lineal, geometría, topología, análisis real y complejo, ecuaciones diferenciales, teoría de grupos, teoría de anillos, teoría de cuerpos, teoría de tipos, teoría de conjuntos, etc.}

\end{document}
