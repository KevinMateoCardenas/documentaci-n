\documentclass{article}
\usepackage[utf8]{inputenc}
\usepackage{amsmath}
\usepackage{amsfonts}

\usepackage{amsmath,amsfonts,amssymb} % Paquetes para matemáticas
\usepackage{graphicx} % Para incluir gráficos
\usepackage{geometry} % Para configurar los márgenes
\usepackage[hidelinks]{hyperref} % Para enlaces hipertextuales sin colores
\usepackage{enumitem} % Para personalizar listas
\usepackage{cite}
\usepackage{graphicx}
\usepackage[utf8]{inputenc}
\usepackage[T1]{fontenc}
\usepackage[spanish]{babel} % Establece el idioma español
\usepackage{pgfplots}
\pgfplotsset{compat=1.16} % Puede que necesites ajustar la versión
\usepackage{csquotes} % Carga el paquete csquotes
\usepackage{graphicx} % Required for inserting images
\usepackage{listings}
\usepackage{xcolor}
\usepackage{hyperref}
\setlength{\parindent}{0.5in}
\usepackage{setspace}
\usepackage{amssymb}
\usepackage{amsthm}
\doublespacing
\renewcommand{\baselinestretch}{1.5}

\theoremstyle{plain}
\newtheorem{proposición}{proposición}[section]
\newtheorem{lema}[proposición]{Lema}
\newtheorem{teo}[proposición]{Teorema}
\newtheorem{coro}[proposición]{Corolario}
\newtheorem{obs}[proposición]{Observación}

\theoremstyle{definition}
\newtheorem{defi}[proposición]{definición}
\newtheorem{ex}[proposición]{Ejercicio}

\newcommand{\Al}{(\mathcal{A},\mathds{F},\odot)}
\newcommand{\A}{\mathcal{A}}
\newcommand{\B}{\mathcal{B}}
\newcommand{\D}{\mathcal{D}}
\newcommand{\C}{\mathbb{C}}
\newcommand{\I}{\mathcal{I}}
\newcommand{\J}{\mathcal{J}}
\newcommand{\R}{\mathbb{R}}
\newcommand{\N}{\mathbb{N}}
\newcommand{\Z}{\mathbb{Z}}
\newcommand{\fu}{f:D\longrightarrow \mathds{R}}
\newcommand{\fun}{f:[a,b]\longrightarrow \mathds{R}}
\newcommand{\E}{\mathcal{E}}
\newcommand{\F}{\mathds{F}}
\newcommand{\op}{``}
\newcommand{\cl}{''}
\newcommand{\po}{^}
\newcommand{\Q}{\matbbb{Q}}
\newcommand{\U}{\mathcal{U}}
\begin{document}

\section*{Solución del examen 3 puntos 1,2,3,5}
\date{24 de octubre 2023}
\begin{teo}
    Sea \( u \in C^2(\mathbb{R} \times (0, \infty)) \) siendo la solución de la ecuación \( u_t = a^2 u_{xx} + bu_x + cu + f(x, t) \) donde \( a, b, c \) son constantes reales y \( f \) es una función dada. Defina la función \( v \) por \( v(x, t) = e^{-ct} u(x - bt, t) \) para \( x \in \mathbb{R} \) y \( t > 0 \). Entonces \( v \) satisface la ecuación no homogénea \( v_t = a^2 v_{xx} + e^{-ct} f(x, t) \).
    
\end{teo}

\begin{ex}
    Resuelve el PVI
    \[
    u_t = a^2 u_{xx} + bu_x + cu + f(x, t), \quad x \in \mathbb{R}, \quad t > 0, \quad u(x, 0) = u_0(x), \quad x \in \mathbb{R}
    \]
    con los siguientes datos:
    \begin{enumerate}
        \item \( f(x, t) = t \sin x, \quad u_0 = 1, \quad a = c > 0, \quad b = 0 \).
        \item \( f(x, t) = h(t) \in C^1([0, \infty)) \) y \( u_0 \) es una función continua acotada.
    \end{enumerate}
\end{ex}




\begin{ex}
     Supongamos que \( u_0 \in C(\mathbb{R}) \) satisface la condición de que \( |u_0(x)| \leq M e^{-\delta|x|} \) para todo \( x \in \mathbb{R} \) y para algunas constantes \( M > 0, \delta > 0 \). Demuestre que la solución \( u \) de la ecuación de calor \( u_t = a^2 u_{xx} \) con dato inicial \( u_0 \) satisface la estimativa
\[
|u(x, t)| \leq M(1 + 4a^2 \delta t)^{-1/2} \exp \left( -\frac{\delta |x|^2}{1 + 4a^2 \delta t} \right)
\]

\end{ex}


\begin{ex}
5. Considere la ecuación de onda \( u_{tt} - c^2 u_{xx} = 0 \), \( x > 0 \), \( t > 0 \) en el primer cuadrante e imponga la siguiente condición de frontera en la frontera \( x = 0 \):

\[
u_t + \alpha u_x = 0, \quad x = 0, \quad t > 0,
\]

y las condiciones iniciales en \( t = 0 \), \( x > 0 \).

\begin{enumerate}
    \item Si \( \alpha \neq c \), derive una fórmula para la solución.
    \item Si \( \alpha = c \), demuestre que no existe una solución en general, pero existe si las condiciones iniciales satisfacen algunas condiciones adicionales. Interprete la condición de frontera en este caso geométricamente.
\end{enumerate}

\end{ex}
\end{document}
