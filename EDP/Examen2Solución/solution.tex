\documentclass{article}
\usepackage[utf8]{inputenc}
\usepackage{amsmath}
\usepackage{amsfonts}

\usepackage{amsmath,amsfonts,amssymb} % Paquetes para matemáticas
\usepackage{graphicx} % Para incluir gráficos
\usepackage{geometry} % Para configurar los márgenes
\usepackage[hidelinks]{hyperref} % Para enlaces hipertextuales sin colores
\usepackage{enumitem} % Para personalizar listas
\usepackage{cite}
\usepackage{graphicx}
\usepackage[utf8]{inputenc}
\usepackage[T1]{fontenc}
\usepackage[spanish]{babel} % Establece el idioma español
\usepackage{pgfplots}
\pgfplotsset{compat=1.16} % Puede que necesites ajustar la versión
\usepackage{csquotes} % Carga el paquete csquotes
\usepackage{graphicx} % Required for inserting images
\usepackage{listings}
\usepackage{xcolor}
\usepackage{hyperref}
\setlength{\parindent}{0.5in}
\usepackage{setspace}
\usepackage{amssymb}
\usepackage{amsthm}
\doublespacing
\renewcommand{\baselinestretch}{1.5}

\theoremstyle{plain}
\newtheorem{proposición}{proposición}
\newtheorem{lema}[proposición]{Lema}
\newtheorem{teo}[proposición]{Teorema}
\newtheorem{coro}[proposición]{Corolario}
\newtheorem{obs}[proposición]{Observación}

\theoremstyle{definition}
\newtheorem{defi}[proposición]{definición}
\newtheorem{ex}[proposición]{Ejercicio}

\newcommand{\Al}{(\mathcal{A},\mathds{F},\odot)}
\newcommand{\A}{\mathcal{A}}
\newcommand{\B}{\mathcal{B}}
\newcommand{\D}{\mathcal{D}}
\newcommand{\C}{\mathbb{C}}
\newcommand{\I}{\mathcal{I}}
\newcommand{\J}{\mathcal{J}}
\newcommand{\R}{\mathbb{R}}
\newcommand{\N}{\mathbb{N}}
\newcommand{\Z}{\mathbb{Z}}
\newcommand{\fu}{f:D\longrightarrow \mathds{R}}
\newcommand{\fun}{f:[a,b]\longrightarrow \mathds{R}}
\newcommand{\E}{\mathcal{E}}
\newcommand{\F}{\mathds{F}}
\newcommand{\op}{``}
\newcommand{\cl}{''}
\newcommand{\po}{^}
\newcommand{\Q}{\matbbb{Q}}
\newcommand{\U}{\mathcal{U}}
\begin{document}
\begin{titlepage}
    \begin{center}
        \vspace*{1cm}
        
        \Huge
        \textbf{Examen 2}
        
        \vspace{0.5cm}
        \LARGE
        Ecuaciones en derivadas parciales
        
        \vspace{1.5cm}
        
        por: \textbf{Kevin Mateo Cárdenas}\\
        profesor: \textbf{Jairo Eloy Castellanos}
        \vfill
        
        \vspace{0.8cm}
        
        \Large
        Facultad de Ciencias Exactas y Naturales\\
        Universidad de Antioquia\\
        2023
    \end{center}
\end{titlepage}
\date{24 de octubre 2023}
\begin{teo}
    Sea \( u \in C^2(\mathbb{R} \times (0, \infty)) \) siendo la solución de la ecuación \( u_t = a^2 u_{xx} + bu_x + cu + f(x, t) \) donde \( a, b, c \) son constantes reales y \( f \) es una función dada. Defina la función \( v \) por \( v(x, t) = e^{-ct} u(x - bt, t) \) para \( x \in \mathbb{R} \) y \( t > 0 \). Entonces \( v \) satisface la ecuación no homogénea \( v_t = a^2 v_{xx} + e^{-ct} f(x-bt, t) \).
    
    \begin{proof}
        Note que si tomamos \(u(\xi,\eta)\), una solución de la ecuación
        \[
            u_\eta = a^2 u_{\xi\xi} + bu_\xi + cu + f(\xi, \eta), \quad \xi \in \mathbb{R}, \quad \eta > 0
        \]
        Entonces, si definimos \(v(x, t) = e^{-ct} u(x - bt, t)\), tenemos que
        \begin{align*}
            v_t &= -ce^{-ct} u(x - bt, t) + e^{-ct} (u_\eta(x - bt, t) - bu_\xi(x - bt, t)) \\
            &= e^{-ct} (u_\eta(x - bt, t) - bu_\xi(x - bt, t) - cu(x - bt, t)) \\
            v_x &= e^{-ct} u_\xi(x - bt, t) \\
            v_{xx} &= e^{-ct} u_{\xi\xi}(x - bt, t)
        \end{align*}
        Y reemplazando en la ecuación, tenemos que
        \begin{align*}
            v_t- a^2 v_{xx} &= e^{-ct} (u_\eta(x - bt, t) - bu_\xi(x - bt, t) - cu(x - bt, t)) - a^2 e^{-ct} u_{\xi\xi}(x - bt, t) \\
            &= e^{-ct} (u_\eta(x - bt, t) - bu_\xi(x - bt, t) - cu(x - bt, t) - a^2 u_{\xi\xi}(x - bt, t)) \\
            &= e^{-ct} f(x - bt, t)
        \end{align*}
    \end{proof}
\end{teo}

\begin{ex}
    Resuelve el PVI
    \begin{align*}
    u_t &= a^2 u_{xx} + bu_x + cu + f(x, t), \quad x \in \mathbb{R}, \quad t > 0\\
    u(x, 0) &= u_0(x), \quad x \in \mathbb{R}
    \end{align*}
    con los siguientes datos:
    \begin{enumerate}
        \item \( f(x, t) = t \sin x, \quad u_0 = 1, \quad a = c > 0, \quad b = 0 \).
        \item \( f(x, t) = h(t) \in C^1([0, \infty)) \) y \( u_0 \) es una función continua acotada.
    \end{enumerate}
    \texttt{Solución:}
    \begin{enumerate}
        \item \( f(x, t) = t \sin x, \quad u_0 = 1, \quad a = c > 0, \quad b = 0 \).\\
            Para este caso, tenemos que \( f(x, t) = t \sin x \), \( u_0 = 1 \), \( a = c > 0 \) y \( b = 0 \). Entonces, por el teorema anterior, tenemos que \( v(x, t) = e^{-ct} u(x, t) \) satisface la ecuación no homogénea \( v_t = a^2 v_{xx} + e^{-ct} f(x, t) \). Entonces, tenemos que $u$ es solución del problema inicial.

            Por lo que basta resolver el PVI
            \begin{align*}
                v_t &= a^2 v_{xx} + e^{-ct} f(x, t), \quad x \in \mathbb{R}, \quad t > 0\\
                v(x, 0) &= e^{-c0} u_0(x) = u_0(x) = 1, \quad x \in \mathbb{R}
            \end{align*}
            
            Este problema se resuelve a partir del principio de Duhamel, es decir, dividimos el problema en dos problemas, (1) y otro (2). Entonces, tenemos que
            \begin{align*}
                v_t &= a^2 v_{xx}, \quad x \in \mathbb{R}, \quad t > s\\
                v(x, s) &= f(x -bt,s), \quad x \in \mathbb{R}
            \end{align*}
            y
            \begin{align*}
                v_t &= a^2 v_{xx}, \quad x \in \mathbb{R}, \quad t > 0\\
                v(x, 0) &= u_0(x) = 1, \quad x \in \mathbb{R}
            \end{align*}
            Entonces, la solución del problema homogéneo (2) es
            \[
                v_{h}(x, t) = \int_{-\infty}^{\infty} \frac{1}{\sqrt{4 \pi a^2 (t - \tau)}} e^{-\frac{(x - \xi)^2}{4a^2 (t - \tau)}} u_{0}(\xi) d\xi
            \]
            y la solución del problema homogéneo (1) es
            \[
                v_{p}(x, t, \tau) = \int_{-\infty}^{\infty} e^{-c(t - \tau)} \frac{1}{\sqrt{4 \pi a^2 (t - \tau)}} e^{-\frac{(x - \xi)^2}{4a^2 (t - \tau)}} \tau \sin \xi d\xi
            \]
            Por lo que la solución del problema es
            \[
                v(x, t) = v_{h}(x, t) + \int_0^t v_{p}(x, t, \tau) d\tau
            \]
            queda
            \begin{align*}
                v(x, t) &= \int_{-\infty}^{\infty} \frac{1}{\sqrt{4 \pi a^2 t}} e^{-\frac{(x - \xi)^2}{4a^2 t}} d\xi\\
                &\hspace{2mm} + \int_0^t \int_{-\infty}^{\infty} e^{-c(t - \tau)} \frac{1}{\sqrt{4 \pi a^2 (t - \tau)}} e^{-\frac{(x - \xi)^2}{4a^2 (t - \tau)}} \tau \sin (\xi-b\tau) d\xi d\tau \\
                &= 1 + \int_0^t \int_{-\infty}^{\infty} e^{-c(t - \tau)} \frac{1}{\sqrt{4 \pi a^2 (t - \tau)}} e^{-\frac{(x - \xi)^2}{4a^2 (t - \tau)}} \tau \sin (\xi-b\tau) d\xi d\tau
            \end{align*}
            entonces la solución del problema inicial es 
            \begin{align*}
                u(x,t) &= e^{ct} v(x + bt,t)\\
                &= e^{ct} \left( 1 + \int_0^t \int_{-\infty}^{\infty} e^{-c(t - \tau)} \frac{1}{\sqrt{4 \pi a^2 (t - \tau)}} e^{-\frac{(x + bt - \xi)^2}{4a^2 (t - \tau)}} \tau \sin (\xi - b\tau) d\xi d\tau \right)
            \end{align*}
        \item \( f(x, t) = h(t) \in C^1([0, \infty)) \) y \( u_0 \) es una función continua acotada.\\
            Para este caso, tenemos que \( f(x, t) = h(t) \in C^1([0, \infty)) \) y \( u_0 \) es una función continua acotada. Entonces, por el teorema anterior, tenemos que \( v(x, t) = e^{-ct} u(x, t) \) satisface la ecuación no homogénea \( v_t = a^2 v_{xx} + e^{-ct} f(x, t) \). Entonces, tenemos que

            Por lo que basta resolver el PVI
            \begin{align*}
                v_t &= a^2 v_{xx} + e^{-ct} h(t), \quad x \in \mathbb{R}, \quad t > 0\\
                v(x, 0) &= e^{-c0} u_0(x) = u_0(x), \quad x \in \mathbb{R}
            \end{align*}
            
            Este problema se resuelve a partir del principio de Duhamel, es decir, dividimos el problema en dos problemas,  (1) y (2). Entonces, tenemos que
            \begin{align*}
                v_t &= a^2 v_{xx}, \quad x \in \mathbb{R}, \quad t > s\\
                v(x, s) &= h(s), \quad x \in \mathbb{R}
            \end{align*}
            y
            \begin{align*}
                v_t &= a^2 v_{xx}, \quad x \in \mathbb{R}, \quad t > 0\\
                v(x, 0) &= u_0(x), \quad x \in \mathbb{R}
            \end{align*}
            Entonces, la solución del problema (2) es
            \[
                v_{h}(x, t) = \int_{-\infty}^{\infty} \frac{1}{\sqrt{4 \pi a^2 (t - \tau)}} e^{-\frac{(x - \xi)^2}{4a^2 (t - \tau)}} u_{0}(\xi) d\xi
            \]
            y la solución del problema (1)
            \[
                v_{p}(x, t. \tau) = \int_{-\infty}^{\infty} e^{-c(t - \tau)} \frac{1}{\sqrt{4 \pi a^2 (t - \tau)}} e^{-\frac{(x - \xi)^2}{4a^2 (t - \tau)}} h(\tau) d\xi
            \]
            Por lo que la solución del problema es
            \[
                v(x, t) = v_{h}(x, t) + \int_0^t v_{p}(x, t, \tau) d\tau
            \]
            queda
            \begin{align*}
                v(x, t) &= \int_{-\infty}^{\infty} \frac{1}{\sqrt{4 \pi a^2 t}} e^{-\frac{(x - \xi)^2}{4a^2 t}} u_0(\xi) d\xi + \int_0^t \int_{-\infty}^{\infty} e^{-c(t - \tau)} \frac{1}{\sqrt{4 \pi a^2 (t - \tau)}} e^{-\frac{(x - \xi)^2}{4a^2 (t - \tau)}} h(\tau) d\xi d\tau \\
            \end{align*}

            Y la solución del problema inicial es
            \begin{align*}
                u(x,t) &= e^{ct} v(x + bt,t)\\
                &= e^{ct} ( \int_{-\infty}^{\infty} \frac{1}{\sqrt{4 \pi a^2 t}} e^{-\frac{(x + bt - \xi)^2}{4a^2 t}} u_0(\xi) d\xi\\
                & +\hspace{2mm} \int_0^t \int_{-\infty}^{\infty} e^{-c(t - \tau)} \frac{1}{\sqrt{4 \pi a^2 (t - \tau)}} e^{-\frac{(x + bt - \xi)^2}{4a^2 (t - \tau)}} h(\tau) d\xi d\tau)
            \end{align*}
            Habría que demostrar que la solución al aplicar el principio de Duhamel es en efecto una solución.

            Si \(v_{h}(x,t,s)\) es solución del problema
            \begin{align*}
                v_t &= a^2 v_{xx}, \quad x \in \mathbb{R}, \quad t > s\\
                v(x, s) &= f(x,s), \quad x \in \mathbb{R}
            \end{align*}
            y \(v_{p}\) es solución del problema
            \begin{align*}
                v_t &= a^2 v_{xx}, \quad x \in \mathbb{R}, \quad t > 0\\
                v(x, 0) &= u_0(x), \quad x \in \mathbb{R}
            \end{align*}
            entonces \(v = v_{p} + \int_0^t v_{h}(x, t, \tau) d\tau\) es solución del problema, pues
            \begin{align*}
                v_t &= \frac{\partial}{\partial t} \left( v_{p} + \int_0^t v_{h}(x,t, \tau) d\tau \right) \\
                &= \frac{\partial}{\partial t} v_{p} + \frac{\partial}{\partial t} \left( \int_0^t v_{h}(x,t, \tau) d\tau \right) \\
                &= \frac{\partial}{\partial t} v_{p} + \int_0^t \frac{\partial v_p(x,t,\tau)}{\partial t} d\tau + v_{h}(x,t,t) \\
                &= \frac{\partial}{\partial t} v_{p} + \int_0^t \frac{\partial v_p(x,t,\tau)}{\partial t} d\tau + f(x,t) \\
            \end{align*}
            y
            \begin{align*}
                v_{xx} &= \frac{\partial}{\partial x} (\frac{\partial}{\partial x} (v_{p} + \int_0^t v_{h}(x,t, \tau) d\tau)) \\
                &= \frac{\partial^2 v_{p}}{\partial x^2} + \frac{\partial^2}{\partial x^2} \int_0^t v_{h}(x,t, \tau) d\tau \\
                &= \frac{\partial^2 v_{p}}{\partial x^2} + \int_0^t \frac{\partial^2 v_{h}(x,t, \tau)}{\partial x^2} d\tau
            \end{align*}
            y al reemplazar en la ecuación, tenemos que
            \begin{align*}
                v_t - a^2 v_{xx} &= \frac{\partial}{\partial t} v_{p} + \int_0^t \frac{\partial v_h(x,t,\tau)}{\partial t} d\tau + f(x,t) - a^2 \left( \frac{\partial^2 v_{p}}{\partial x^2} + \int_0^t \frac{\partial^2 v_{h}(x,t, \tau)}{\partial x^2} d\tau \right) \\
                &= \frac{\partial}{\partial t} v_{p} - a^2\frac{\partial^2 v_{p}}{\partial x^2} + \int_0^t \frac{\partial v_h(x,t,\tau)}{\partial t} -a^2\frac{\partial^2 v_{h}(x,t, \tau)}{\partial x^2} d\tau + f(x,t)\\
                &= f(x,t)
            \end{align*}
            y la condición inicial se cumple, pues
            \begin{align*}
                v(x,0) &= v_{p}(x,0) + \int_0^0 v_{h}(x,0,\tau) d\tau \\
                &= u_0(x)
            \end{align*}
    \end{enumerate}
\end{ex}

\newpage
\begin{ex}
     Supongamos que \( u_0 \in C(\mathbb{R}) \) satisface la condición de que \( |u_0(x)| \leq M e^{-\delta|x|^{2}} \) para todo \( x \in \mathbb{R} \) y para algunas constantes \( M > 0, \delta > 0 \). Demuestre que la solución \( u \) de la ecuación de calor \( u_t = a^2 u_{xx} \) con dato inicial \( u_0 \) satisface la estimativa
\[
|u(x, t)| \leq M(1 + 4a^2 \delta t)^{-1/2} \exp \left( -\frac{\delta |x|^2}{1 + 4a^2 \delta t} \right)
\]
\texttt{Solución:}

Tenemos el problema
\begin{align*}
    u_t &= a^2 u_{xx}, \quad x \in \mathbb{R}, \quad t > 0\\
    u(x, 0) &= u_0(x), \quad x \in \mathbb{R}
\end{align*}
Donde \(u_0 \in C(\mathbb{R})\) satisface la condición de que \( |u_0(x)| \leq M e^{-\delta|x|^{2}} \) para todo \( x \in \mathbb{R} \) y para algunas constantes \( M > 0, \delta > 0 \).

Sabemos que la solución viene dada por
\[
    u(x,t) = \int_{-\infty}^{\infty} \frac{1}{\sqrt{4 \pi a^2 t}} e^{-\frac{(x - \xi)^2}{4a^2 t}} u_0(\xi) d\xi
\]
y si \(|u_0(x)| \leq M e^{-\delta|x|^{2}}\), entonces
\begin{align*}
    |u(x,t)| &= \left| \int_{-\infty}^{\infty} \frac{1}{\sqrt{4 \pi a^2 t}} e^{-\frac{(x - \xi)^2}{4a^2 t}} u_0(\xi) d\xi \right| \\
    &\leq \int_{-\infty}^{\infty} \frac{1}{\sqrt{4 \pi a^2 t}} e^{-\frac{(x - \xi)^2}{4a^2 t}} |u_0(\xi)| d\xi \\
    &\leq \int_{-\infty}^{\infty} \frac{1}{\sqrt{4 \pi a^2 t}} e^{-\frac{(x - \xi)^2}{4a^2 t}} M e^{-\delta|\xi|^{2}} d\xi \\
    &= \frac{M}{2a\sqrt{\pi t}} \int_{-\infty}^{\infty} e^{-\frac{(x - \xi)^2}{4a^2 t} -\delta|\xi|^{2}} d\xi\\
\end{align*}

Hagamos por aparte la integral
\begin{align*}
    \int_{-\infty}^{\infty} e^{-\frac{(x - \xi)^2}{4a^2 t} -\delta|\xi|^{2}} d\xi &= \int_{-\infty}^{0} e^{-\frac{(x - \xi)^2}{4a^2 t}} e^{-\delta|\xi|} d\xi + \int_{0}^{\infty} e^{-\frac{(x - \xi)^2}{4a^2 t}} e^{-\delta|\xi|^{2}} d\xi \\
    &= \int_{0}^{\infty} e^{-\frac{(x - \xi)^2}{4a^2 t}} e^{-\delta|\xi|^{2}} d\xi + \int_{0}^{\infty} e^{-\frac{(x - \xi)^2}{4a^2 t}} e^{-\delta|\xi|^{2}} d\xi \\
    &= 2 \int_{0}^{\infty} e^{-\frac{(x - \xi)^2}{4a^2 t}-\delta\xi^{2}} d\xi \\
\end{align*}
 Por otra parte

\begin{align*}
    \frac{(x - \xi)^2}{4a^2 t} + \delta\xi^{2} &= \frac{x^2 - 2x\xi + \xi^2}{4a^2 t} + \delta\xi^{2} \\
    &= \frac{x^2 - 2x\xi + \xi^2 + 4a^2 t \delta \xi^2}{4a^2 t} \\
    &= \frac{x^2 - 2x\xi + (4a^2\delta  t + 1)\xi^2}{4a^2 t}
\end{align*}
Completando cuadrados tenemos
\begin{align*}
    \frac{(x - \xi)^2}{4a^2 t} + \delta\xi^{2} &= \frac{x^2 - 2x\xi + (4a^2\delta  t + 1)\xi^2}{4a^2 t}\\
    &= \frac{x^2 - 2x\xi + (4a^2\delta  t + 1)\xi^2 +x^2(1+4a^2\delta  t)^{-1} - x^2(1+4a^2\delta  t)^{-1}}{4a^2 t}\\
    &= \frac{(x(1 + 4a^2 \delta  t)^{-1/2} - \xi(1+4a^2\delta  t)^{1/2})^{2}}{4a^2t} - \frac{x^2(1+4a^2\delta  t)^{-1} - x^{2}}{4a^{2}t}
\end{align*}
Además

\begin{align*}
    \frac{x^2(1+a^2\delta  t)^{-1} - x^{2}}{4a^{2}t} &= \frac{\frac{x^2}{1+4a^2\delta  t} - x^{2}}{4a^{2}t} \\
    &= \frac{x^2 - x^{2}(1+4a^2\delta  t)}{4a^{2}t(1+4a^2\delta  t)} \\
    &= \frac{x^2(4a^2\delta  t)}{4a^{2}t(1+4a^2\delta  t)} \\
    &= \frac{x^2\delta}{1+4a^2\delta  t}
\end{align*}
Por lo tanto

\begin{align*}
    |u(x,t)| &= \frac{M}{2a\sqrt{\pi t}} \int_{-\infty}^{\infty} e^{-\frac{(x - \xi)^2}{4a^2 t} -\delta|\xi|^{2}} d\xi\\
    &= \frac{M}{2a\sqrt{\pi t}} \int_{-\infty}^{\infty} e^{-\frac{(x(1 + 4a^2 \delta  t)^{-1/2} - \xi(1+4a^2\delta  t)^{1/2})^{2}}{4a^2t} + \frac{x^2(1+4a^2\delta  t)^{-1} + x^{2}}{4a^{2}t}} d\xi \\
    &= \frac{M}{2a\sqrt{\pi t}} \int_{-\infty}^{\infty} e^{-\frac{(x(1 + 4a^2 \delta  t)^{-1/2} - \xi(1+4a^2\delta  t)^{1/2})^{2}}{4a^2t}} e^{-\frac{x^2\delta}{1+4a^2\delta  t}} d\xi \\
    &= \frac{M}{2a\sqrt{\pi t}} e^{-\frac{x^2\delta}{1+4a^2\delta  t}} \int_{-\infty}^{\infty} e^{-\frac{(x(1 + 4a^2 \delta  t)^{-1/2} - \xi(1+4a^2\delta  t)^{1/2})^{2}}{4a^2t}} d\xi \\
\end{align*}
Consideremos el siguiente cambio de variable
\[
    \eta = \frac{x(1 + 4a^2 \delta  t)^{1/2} - \xi(1+4a^2\delta  t)^{1/2}}{2a\sqrt{t}}
\]
Entonces
\begin{align*}
    d\eta &= \frac{(1 + 4a^2 \delta  t)^{1/2}}{2a\sqrt{t}} d\xi \\
    |u(x,t)| &\leq \frac{M}{(1 + 4a^2 \delta  t)^{1/2}\sqrt{\pi}} e^{-\frac{x^2\delta}{1+4a^2\delta  t}} \int_{-\infty}^{\infty} e^{-\eta^{2}} d\eta \\
    &= \frac{M}{(1 + 4a^2 \delta  t)^{1/2}\sqrt{\pi}} e^{-\frac{x^2\delta}{1+4a^2\delta  t}} \sqrt{\pi} \\
    &= \frac{M}{(1 + 4a^2 \delta  t)^{1/2}}e^{-\frac{x^2\delta}{1+4a^2\delta  t}}
\end{align*}
\end{ex}

\newpage
\begin{ex}
Considere la ecuación de onda \( u_{tt} - c^2 u_{xx} = 0 \), \( x > 0 \), \( t > 0 \) en el primer cuadrante e imponga la siguiente condición de frontera en la frontera \( x = 0 \):

\[
u_t + \alpha u_x = 0, \quad x = 0, \quad t > 0,
\]

y las condiciones iniciales en \( t = 0 \), \( x > 0 \).

\begin{enumerate}
    \item Si \( \alpha \neq c \), derive una fórmula para la solución.
    \item Si \( \alpha = c \), demuestre que no existe una solución en general, pero existe si las condiciones iniciales satisfacen algunas condiciones adicionales. Interprete la condición de frontera en este caso geométricamente.
\end{enumerate}
\texttt{Solución:}
Tenemos el problema
\begin{align*}
    u_{tt} - c^2 u_{xx} &= 0, \quad x > 0, \quad t > 0\\
    u_t(0,t) + \alpha u_x(0,t) &= 0, \quad x = 0, \quad t > 0\\
    u(x, 0) &= f(x), \quad x \geq 0,\\
    u_t(x, 0) &= g(x), \quad x > 0
\end{align*}

Si \(x_0 \geq ct_0\), de la formula de d'Alembert pues $x_0 - ct_0 \geq 0$ y $x_0 + ct_0 \geq 0$, tenemos que
\begin{align*}
    u(x_0,t_0) &= \frac{1}{2} \left( f(x_0 - ct_0) + f(x_0 + ct_0) \right) + \frac{1}{2c} \int_{x_0 - ct_0}^{x_0 + ct_0} g(\xi) d\xi
\end{align*}
\begin{align*}
    F(\tau) &= \frac{f(\tau)}{2} - \frac{1}{2c} \int_{0}^{\tau} g(\xi) d\xi \quad \tau \geq 0\\
    G(\tau) &= \frac{f(\tau)}{2} + \frac{1}{2c} \int_{0}^{\tau} g(\xi) d\xi \quad \tau \geq 0
\end{align*}

Por otra parte si \(-ct_0 < x_0 < ct_0\), tenemos que $x_0 - ct_0 < 0$ y $x_0 + ct_0 > 0$, por lo que
\begin{align*}
    u(x_0,t_0) &= F(x_0-ct_0) + G(x_0+ct_0)
\end{align*}
Derivando respecto a \(t\) y a \(x\), tenemos que
\begin{align*}
    \frac{\partial u}{\partial t}(x,t) &= -cF'(x-ct) + cG'(x+ct)\\
    \frac{\partial u}{\partial x}(x,t) &= F'(x-ct) + G'(x+ct)
\end{align*}
Por lo que
\begin{align*}
    u_t(0,t) + \alpha u_x(0,t) &= cG'(ct) - cF'(-ct) + \alpha F'(-ct) + \alpha G'(ct)\\
    &= 0
\end{align*}
Equivalentemente
\[
    (c + \alpha )G'(ct) + (\alpha -c)F'(-ct) = 0
\]
Es decir
\[
    G'(ct) = \frac{c -\alpha}{c + \alpha }F'(-ct)
\]
Integrando
\[
    -(G(ct) - G(0)) = \frac{c -\alpha}{c + \alpha }(F(-ct)-F(0))
\]
Por lo tanto 
\begin{align*}
    F(-ct) &= \frac{c +\alpha}{c - \alpha }(G(0) - G(ct)) + F(0)\\
    &= \frac{c +\alpha}{c - \alpha }\left(\frac{f(0)}{2} - \frac{f(ct)}{2} - \frac{1}{2c} \int_{0}^{ct} g(\xi) d\xi\right) + \frac{f(0)}{2}
\end{align*}
Por lo tanto si $\tau > 0$
\begin{align*}
    F(-\tau) &= \frac{c + \alpha}{c - \alpha }\left(\frac{f(0)}{2} - \frac{f(\tau)}{2} - \frac{1}{2c} \int_{0}^{\tau} g(\xi) d\xi\right) + \frac{f(0)}{2}\\
\end{align*}
Entonces si \(-ct_0<x_0 < ct_0\), tenemos que
\begin{align*}
    u(x_0,t_0) &= F(x_0-ct_0) + G(x_0+ct_0)\\
    &= \frac{c + \alpha}{c - \alpha }\left(\frac{f(0)}{2} - \frac{f(-x_0+ct_0)}{2} - \frac{1}{2c} \int_{0}^{x_0-ct_0} g(\xi) d\xi\right) + \frac{f(0)}{2}\\
    &\hspace{2mm} + \frac{f(x_0+ct_0)}{2} + \frac{1}{2c} \int_{0}^{x_0+ct_0} g(\xi) d\xi
\end{align*}

Por lo tanto, si $\alpha \neq c$, $u$ se define por partes de la siguiente manera

\begin{align*}
    u(x,t) &= \frac{1}{2}(f(x - ct) + f(x + ct)) + \frac{1}{2c} \int_{x - ct}^{x + ct} g(\xi) d\xi \hspace{45mm}  x \geq ct\\
    u(x,t)&=\frac{c +\alpha}{c - \alpha }\left(\frac{f(0)}{2} - \frac{f(-x+ct)}{2} - \frac{1}{2c} \int_{0}^{x-ct} g(\xi) d\xi\right)\\
    &\hspace{2mm} + \frac{f(0)}{2} + \frac{f(x+ct)}{2} + \frac{1}{2c} \int_{0}^{x+ct} g(\xi) d\xi \hspace{45mm}  0 \leq x < ct\\
\end{align*}
Por otra parte, fisicamente hablando, la condición inicial
\[
    \frac{\partial u}{\partial t}(0,t) + \alpha \frac{\partial u}{\partial x}(0,t) = 0 
\]
Es equivalente a
\begin{align*}
    \frac{\partial u}{\partial x}\frac{\partial x}{\partial t} = -\alpha \frac{\partial u}{\partial x}\\
\end{align*}
Es decir, que
\[
    \frac{\partial x}{\partial t} = -\alpha
\]
Lo que podemos interpretar como que la frontera se mueve con velocidad $-\alpha$.

Notemos que $c = \alpha$, implica $G$ constante, dado que
\begin{align*}
    G'(ct) &= \frac{c -\alpha}{c + \alpha }F'(-ct)\\
    &= 0
\end{align*}
Pero recordemos que
\begin{align*}
    G(\tau) &= \frac{f(\tau)}{2} + \frac{1}{2c} \int_{0}^{\tau} g(\xi) d\xi \quad \tau \geq 0\\
    F(\tau) &= \frac{f(\tau)}{2} - \frac{1}{2c} \int_{0}^{\tau} g(\xi) d\xi \quad \tau \geq 0
\end{align*}
Y $G$ constante implica que
\begin{align*}
    \frac{f(\tau)}{2} + \frac{1}{2c} \int_{0}^{\tau} g(\xi) d\xi &= \frac{f(0)}{2} + \frac{1}{2c} \int_{0}^{0} g(\xi) d\xi\\
    \frac{1}{2c} \int_{0}^{\tau} g(\xi) d\xi &= \frac{f(0)}{2} -\frac{f(\tau)}{2}\\
\end{align*}
reemplazando en $F$ tenemos que
\begin{align*}
    F(\tau) &= \frac{f(\tau)}{2} - \frac{1}{2c} \int_{0}^{\tau} g(\xi) d\xi\\
    &= \frac{f(\tau)}{2} - \frac{f(0)}{2} + \frac{f(\tau)}{2}\\
    &= f(\tau) - \frac{f(0)}{2}
\end{align*}
y reemplazando en $u$
\begin{align*}
    u(x,t) &= F(x - ct) + G(x + ct) = f(x - ct) - \frac{f(0)}{2} + \frac{f(0)}{2}\\
    &= f(x - ct)
\end{align*}
En particular, derivando respecto a $t$ tenemos que
\begin{align*}
    u_t(x,t) &= -cf'(x - ct)\\
    u_t(x,0) &= g(x) = -cf'(x)
\end{align*}
Lo cual genera una condición adicional sobre $f$ y $g$, es decir, que $g(x) = -cf'(x)$, es decir, que $f$ es de la forma $f(x) = -\frac{1}{c} \int_{0}^{x} g(\xi) d\xi + f(0)$. Por otra parte, geometricamente hablando, esto signufica que $u$ solo depende de $x - ct$, es decir, que la solución se propaga con rapidez $c$ en sentido contrario.
\end{ex}
\end{document}
